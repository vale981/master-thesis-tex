% Gemini theme
% https://github.com/anishathalye/gemini

\documentclass[final]{beamer}

\pdfvariable suppressoptionalinfo 512\relax

% ====================
% Packages
% ====================

\usepackage[T1]{fontenc}
\usepackage{../hiromacros}
\usepackage{fontspec}
\usepackage{unicode-math}
\setmathfont{texgyrepagella-math.otf}

\usepackage{lmodern}
\usepackage[size=A0,orientation=portrait,scale=1.0]{beamerposter}
\usetheme{gemini}
\usecolortheme{mit}
\usepackage{graphicx}
\usepackage{booktabs}
\usepackage{tikz}
\usepackage{pgfplots}
\pgfplotsset{compat=1.14}

\graphicspath{{figs/}}

% ====================
% Lengths
% ====================

% If you have N columns, choose \sepwidth and \colwidth such that
% (N+1)*\sepwidth + N*\colwidth = \paperwidth
\newlength{\sepwidth}
\newlength{\colwidth}
\setlength{\sepwidth}{0.066\paperwidth}
\setlength{\colwidth}{0.4\paperwidth}

\newcommand{\separatorcolumn}{\begin{column}{\sepwidth}\end{column}}

% ====================
% Title
% ====================

\title{Calculating Energy Flows in Strongly Coupled Open Quantum
  Systems}

\author{\underline{Valentin Boettcher}\inst{1}, Richard Hartmann\inst{1},
  Konstantin Beyer\inst{1}, Walter Strunz\inst{1}}

\institute[shortinst]{\inst{1} Institute for Theoretical Physics, Dresden, Germany}

% ====================
% Footer (optional)
% ====================

\footercontent{
  \href{https://tu-dresden.de/mn/physik/itp/tqo/die-professur}{https://tu-dresden.de/mn/physik/itp/tqo/die-professur} \hfill
  Group for Theoretical Quantum Optics, TU-Dresden \hfill
  \href{mailto:valentin.boettcher@tu-dresden.de}{valentin.boettcher@tu-dresden.de}}
% (can be left out to remove footer)

% ====================
% Logo (optional)
% ====================

% use this to include logos on the left and/or right side of the header:
\logoleft{\includegraphics[height=3cm]{Logo_TU_Dresden.pdf}}

% ====================
% Body
% ====================

\begin{document}

\begin{frame}[t]
\begin{columns}[t]
\separatorcolumn

\begin{column}{\colwidth}
  \begin{block}{A block title}

    Some block contents, followed by a diagram, followed by a dummy paragraph.

    \begin{figure}
      \centering
      \begin{tikzpicture}[scale=6]
        \draw[step=0.25cm,color=gray] (-1,-1) grid (1,1);
        \draw (1,0) -- (0.2,0.2) -- (0,1) -- (-0.2,0.2) -- (-1,0)
          -- (-0.2,-0.2) -- (0,-1) -- (0.2,-0.2) -- cycle;
      \end{tikzpicture}
      \caption{A figure caption.}
    \end{figure}

    Lorem ipsum dolor sit amet, consectetur adipiscing elit. Morbi ultricies
    eget libero ac ullamcorper. Integer et euismod ante. Aenean vestibulum
    lobortis augue, ut lobortis turpis rhoncus sed. Proin feugiat nibh a
    lacinia dignissim. Proin scelerisque, risus eget tempor fermentum, ex
    turpis condimentum urna, quis malesuada sapien arcu eu purus.

  \end{block}

  \begin{block}{A block containing a list}

    Nam vulputate nunc felis, non condimentum lacus porta ultrices. Nullam sed
    sagittis metus. Etiam consectetur gravida urna quis suscipit.

    \begin{itemize}
      \item \textbf{Mauris tempor} risus nulla, sed ornare
      \item \textbf{Libero tincidunt} a duis congue vitae
      \item \textbf{Dui ac pretium} morbi justo neque, ullamcorper
    \end{itemize}

    Eget augue porta, bibendum venenatis tortor.

  \end{block}

  \begin{alertblock}{A highlighted block}

    This block catches your eye, so \textbf{important stuff} should probably go
    here.

    Curabitur eu libero vehicula, cursus est fringilla, luctus est. Morbi
    consectetur mauris quam, at finibus elit auctor ac. Aliquam erat volutpat.
    Aenean at nisl ut ex ullamcorper eleifend et eu augue. Aenean quis velit
    tristique odio convallis ultrices a ac odio.

    \begin{itemize}
      \item \textbf{Fusce dapibus tellus} vel tellus semper finibus. In
        consequat, nibh sed mattis luctus, augue diam fermentum lectus.
      \item \textbf{In euismod erat metus} non ex. Vestibulum luctus augue in
        mi condimentum, at sollicitudin lorem viverra.
      \item \textbf{Suspendisse vulputate} mauris vel placerat consectetur.
        Mauris semper, purus ac hendrerit molestie, elit mi dignissim odio, in
        suscipit felis sapien vel ex.
    \end{itemize}

    Aenean tincidunt risus eros, at gravida lorem sagittis vel. Vestibulum ante
    ipsum primis in faucibus orci luctus et ultrices posuere cubilia Curae.

  \end{alertblock}

\end{column}

\separatorcolumn

\begin{column}{\colwidth}

  \begin{block}{A block containing an enumerated list}

    Vivamus congue volutpat elit non semper. Praesent molestie nec erat ac
    interdum. In quis suscipit erat. \textbf{Phasellus mauris felis, molestie
    ac pharetra quis}, tempus nec ante. Donec finibus ante vel purus mollis
    fermentum. Sed felis mi, pharetra eget nibh a, feugiat eleifend dolor. Nam
    mollis condimentum purus quis sodales. Nullam eu felis eu nulla eleifend
    bibendum nec eu lorem. Vivamus felis velit, volutpat ut facilisis ac,
    commodo in metus.

    \begin{enumerate}
      \item \textbf{Morbi mauris purus}, egestas at vehicula et, convallis
        accumsan orci. Orci varius natoque penatibus et magnis dis parturient
        montes, nascetur ridiculus mus.
      \item \textbf{Cras vehicula blandit urna ut maximus}. Aliquam blandit nec
        massa ac sollicitudin. Curabitur cursus, metus nec imperdiet bibendum,
        velit lectus faucibus dolor, quis gravida metus mauris gravida turpis.
      \item \textbf{Vestibulum et massa diam}. Phasellus fermentum augue non
        nulla accumsan, non rhoncus lectus condimentum.
    \end{enumerate}

  \end{block}

  \begin{block}{Fusce aliquam magna velit}

    Et rutrum ex euismod vel. Pellentesque ultricies, velit in fermentum
    vestibulum, lectus nisi pretium nibh, sit amet aliquam lectus augue vel
    velit. Suspendisse rhoncus massa porttitor augue feugiat molestie. Sed
    molestie ut orci nec malesuada. Sed ultricies feugiat est fringilla
    posuere.

    \begin{figure}
      \centering
      \begin{tikzpicture}
        \begin{axis}[
            scale only axis,
            no markers,
            domain=0:2*pi,
            samples=100,
            axis lines=center,
            axis line style={-},
            ticks=none]
          \addplot[red] {sin(deg(x))};
          \addplot[blue] {cos(deg(x))};
        \end{axis}
      \end{tikzpicture}
      \caption{Another figure caption.}
    \end{figure}

  \end{block}

  \begin{block}{Nam cursus consequat egestas}

    Nulla eget sem quam. Ut aliquam volutpat nisi vestibulum convallis. Nunc a
    lectus et eros facilisis hendrerit eu non urna. Interdum et malesuada fames
    ac ante \textit{ipsum primis} in faucibus. Etiam sit amet velit eget sem
    euismod tristique. Praesent enim erat, porta vel mattis sed, pharetra sed
    ipsum. Morbi commodo condimentum massa, \textit{tempus venenatis} massa
    hendrerit quis. Maecenas sed porta est. Praesent mollis interdum lectus,
    sit amet sollicitudin risus tincidunt non.

    Etiam sit amet tempus lorem, aliquet condimentum velit. Donec et nibh
    consequat, sagittis ex eget, dictum orci. Etiam quis semper ante. Ut eu
    mauris purus. Proin nec consectetur ligula. Mauris pretium molestie
    ullamcorper. Integer nisi neque, aliquet et odio non, sagittis porta justo.

    \begin{itemize}
      \item \textbf{Sed consequat} id ante vel efficitur. Praesent congue massa
        sed est scelerisque, elementum mollis augue iaculis.
        \begin{itemize}
          \item In sed est finibus, vulputate
            nunc gravida, pulvinar lorem. In maximus nunc dolor, sed auctor eros
            porttitor quis.
          \item Fusce ornare dignissim nisi. Nam sit amet risus vel lacus
            tempor tincidunt eu a arcu.
          \item Donec rhoncus vestibulum erat, quis aliquam leo
            gravida egestas.
        \end{itemize}
      \item \textbf{Sed luctus, elit sit amet} dictum maximus, diam dolor
        faucibus purus, sed lobortis justo erat id turpis.
      \item \textbf{Pellentesque facilisis dolor in leo} bibendum congue.
        Maecenas congue finibus justo, vitae eleifend urna facilisis at.
    \end{itemize}

  \end{block}
\end{column}

\separatorcolumn
\end{columns}
\end{frame}

\end{document}
