% Gemini theme
% https://github.com/anishathalye/gemini

\documentclass[final]{beamer}

\pdfvariable suppressoptionalinfo 512\relax

% ====================
% Packages
% ====================

\usepackage[T1]{fontenc}
\usepackage{../hiromacros}

\usepackage{lmodern}
\usepackage[size=A1,orientation=portrait,scale=1.0]{beamerposter}
\usetheme{gemini}
\usecolortheme{mit}
\usepackage{graphicx}
\usepackage{booktabs}
\usepackage{tikz}
\usepackage{pgfplots}
\pgfplotsset{compat=1.14}
\usepackage{physics}
\usepackage{cleveref}


\graphicspath{{figs/}}

% Bibliographic Stuff
\usepackage[backend=biber, language=english, style=phys]{biblatex}
\addbibresource{references.bib}

\DeclareDocumentCommand\vectorbold{ s m
}{\IfBooleanTF{#1}{\symbfit{#2}}{\symbf{#2}}} % Vector bold [star for
% Greek and italic Roman]

% ====================
% Lengths
% ====================

% If you have N columns, choose \sepwidth and \colwidth such that
% (N+1)*\sepwidth + N*\colwidth = \paperwidth
\newlength{\sepwidth}
\newlength{\colwidth}
\setlength{\sepwidth}{0.066\paperwidth}
\setlength{\colwidth}{0.4\paperwidth}

\newcommand{\separatorcolumn}{\begin{column}{\sepwidth}\end{column}}

% ====================
% Title
% ====================

\title{Calculating Energy Flows in Strongly Coupled Open Quantum
  Systems with HOPS}

\author{\underline{Valentin Boettcher}\inst{1}, Richard Hartmann\inst{1},
  Konstantin Beyer\inst{1}, Walter Strunz\inst{1}}

\institute[shortinst]{\inst{1} Institute for Theoretical Physics, Dresden, Germany}

% ====================
% Footer (optional)
% ====================

\footercontent{
  \href{https://tu-dresden.de/mn/physik/itp/tqo/die-professur}{https://tu-dresden.de/mn/physik/itp/tqo/die-professur} \hfill
  Group for Theoretical Quantum Optics, TU-Dresden \hfill
  \href{mailto:valentin.boettcher@tu-dresden.de}{valentin.boettcher@tu-dresden.de}}
% (can be left out to remove footer)

% ====================
% Logo (optional)
% ====================

% use this to include logos on the left and/or right side of the header:
\logoleft{\includegraphics[height=3cm]{Logo_TU_Dresden.pdf}}

% ====================
% Body
% ====================

\begin{document}

\begin{frame}[t]
\begin{columns}[t]
\separatorcolumn

\begin{column}{\colwidth}
  \begin{block}{Premise}
    \begin{itemize}
    \item Application of thermodynamic notions to strongly coupled and
      non-Markovian quantum systems is non-trivial
      (\cite{Rivas2019Oct,Kato2016Dec,Strasberg2021Aug,Talkner2020Oct} and many
      more)
    \item Dynamics of bath and interaction hamiltonians plays an
      important role \(\rightarrow\) must not be neglected in the
      strong coupling regime
    \item The ``Hierarchy of Pure States''
      (HOPS~\cite{Hartmann2017Dec,Diosi1998Mar}) gives us the ability
      to simulate non-Markovian and strongly coupled open quantum
      systems exactly in a scalable way.
    \item Because HOPS simulates global dynamics \(\rightarrow\) gives
      access to certain bath dynamics \emph{with no additional effort}
    \end{itemize}
  \end{block}
  \begin{block}{NMQSD/HOPS}
    Consider the model of a general quantum system (\(H_\sys(t)\))
    coupled to \(N\) baths
    \begin{equation}
      \label{eq:generalmodel}
      H(t) = H_\sys(t) + ∑_{n=1}^N \qty[L_n^†(t)B_n + \hc] + ∑_{n=1}^NH_B\nth ,
    \end{equation}
    with \(B_n=∑_{λ} g_λ\nth a_λ\nth\) and
    \(H_B\nth=∑_λω_λ\nth \qty(b_λ\nth)^\dag b_λ\nth\).  Projecting
    onto coherent bath states
    \begin{equation}
      \label{eq:projected}
      \ket{ψ(t)} = ∫∏_{n=1}^N{\qty(\frac{\dd{\vb{z}\nth}}{π^{N_n}}\eu^{-\abs{\vb{z}}^2})}\ket{ψ(t,\underline{\vb{z}}^\ast)}\ket{\underline{\vb{z}}}
    \end{equation}
    leads to \emph{stochastic} Non-Markovian
    Quantum State Diffusion (NMQSD)
    \begin{equation}
      \label{eq:nmqsd}
      ∂_tψ_t(\vb{η}^\ast_t) = -\iu H(t) ψ_t(\vb{η}^\ast_t) +
      \vb{L}\cdot \vb{η}^\ast_tψ_t(\vb{η}^\ast_t) - ∑_{n=1}^N L(t)_n^†∫_0^t\dd{s}α_n(t-s)\fdv{ψ_t(\vb{η}^\ast_t)}{η^\ast_n(s)},
    \end{equation}
    where the
    \(α_n(τ) = \ev{B_n(t) B_n(0)} = ∑_λ\abs{g_λ}^2 \eu^{-\iu ω_λ t}\)
    {\tiny (interaction picture)} are the bath correlation functions
    and the \(η_n=(\vb{η})_n\) are complex valued Gaussian processes
    with \(\mathcal{M}(η_n(t))=\mathcal{M}(η_n(t)η_n(s))=0\) and
    \(\mathcal{M}(η_n(t)η_n^\ast(s))=α_n(t-s)\). The reduced state of
    the system is recovered through
    \(ρ=\mathcal{M}(ψ_t(\vb{η}^\ast_t)ψ_t^\dag(\vb{η}^\ast_t))\).

    With \(α_n(τ)=∑_{\mu}^{M_n}=G_μ\nth\eu^{-W_μ\nth τ}\) we define
    \begin{align}
      \label{eq:dop}
      D_μ\nth(t) &\equiv ∫_0^t\dd{s}G_μ\nth\eu^{-W_μ\nth
                   (t-s)}\fdv{η^\ast_n(s)} &
      D^{\underline{\vb{k}}} &\equiv
      ∏_{n=1}^N∏_{μ=1}^{M_n}
      {\sqrt{\frac{\underline{\vb{k}}_{n,μ}!}{\qty(G\nth_μ)^{\underline{\vb{k}}_{n,μ}}}}
                               \frac{1}{\iu^{\underline{\vb{k}}_{n,μ}}}}\qty(D_μ\nth)^{\underline{\vb{k}}_{n,μ}}\\
        ψ^{\underline{\vb{k}}} &\equiv D^{\underline{\vb{k}}}ψ \equiv \braket{\kmat}{Ψ}.
    \end{align}
    For the Fock-space embedded hierarchy state \(\ket{Ψ}\) we find
    \begin{equation}
      \label{eq:fockhops}
      \begin{aligned}
        ∂_t\ket{Ψ} &= \qty[
                     \begin{aligned}
                       -\iu H_\sys + \vb{L}\cdot\vb{η}^\ast &-
                                                              ∑_{n=1}^N∑_{μ=1}^{M_n}b_{n,μ}^\dag b_{n,μ} W\nth_μ \\
                                                            &\qquad+
                                                              \iu ∑_{n=1}^N∑_{μ=1}^{M_n} \sqrt{G_{n,μ}} \qty(b^†_{n,μ}L_n +
                                                              b_{n,μ}L^†_n)
                     \end{aligned}
                     ] \ket{Ψ}.
      \end{aligned}
    \end{equation}
    Truncating the hierarchy depth \(\kmat\) in \cref{eq:fockhops}
    yields the numeric method.

    Finite temperature can be dealt with through substituting
    \(B(t)\rightarrow B(t)+ξ(t)\) with
    \begin{equation}
      \label{eq:thermproc}
      \begin{aligned}
        \mathcal{M}(ξ(t))&=0=\mathcal{M}(ξ(t) ξ(s)) \\
        \mathcal{M}\left(ξ(t) ξ^{*}(s)\right)&=\frac{1}{\pi} ∫_{0}^{∞}
                                               \dd{ω} \bar{n}(\beta ω) J(ω) e^{-{\iu} ω(t-s)} \\
        J(ω)&=π\sum_λ\abs{g_λ}^2δ(w-ω_λ).
      \end{aligned}
    \end{equation}
    See~\cite{Hartmann2017Dec} for details about finite temperatures
    and the nonlinear method.
  \end{block}

  \begin{alertblock}{Bath Observables}
    From \cref{eq:nmqsd,eq:dop} we find the correspondence \(B(t)
    \leftrightarrow D_t \leftrightarrow ψ^\kmat\) \(\implies\) can
    calulate observables of type \(O_\sys\otimes (B^a)^\dag B^b\) and
    time derivatives thereof. This grants the hierarchy states a
    utility beyond the mere simulation of reduced dynamics.

    \begin{description}
    \item[Bath Energy Flow] We consider the zero temperature and
      one-bath case.
      \begin{equation}
        \label{eq:heatflowdef}
        \begin{aligned}
          J &= - \dv{\ev{H_\bath}}{t}  = \ev{L(t)^†∂_t B(t) + L(t)∂_t
              B^†(t)}_\inter \\
            &=-\i \mathcal{M}_{η^\ast}\bra{\psi(η,
              t)}L(t)^†\dot{D}_t\ket{\psi(η^\ast,t)} + \cc\\
            &= - ∑_\mu\sqrt{G_\mu}W_\mu
              \mathcal{M}_{η^\ast}\bra{\psi^{(0)}(η,t)}L(t)^†\ket{\psi^{\vb{e}_\mu}(η^\ast,t)} + \cc
        \end{aligned}
      \end{equation}
      Thus, the expectation value of the bath energy flow is connected
      to the first hierarchy level states in a transparent and easy to
      calculate manner.
    \item[Interaction Energy]
      A similar expression may be found for the expectation value of
      the interaction energy
      \begin{equation}
        \label{eq:intexp}
        \begin{aligned}
          \ev{H_\inter} &=-\i \mathcal{M}_{η^\ast}\bra{\psi(η,
          t)}L(t)^†D_t\ket{\psi(η^\ast,t)} + \cc \\
                        &=  ∑_\mu\sqrt{G_\mu}
                          \mathcal{M}_{η^\ast}\bra{\psi^{(0)}(η,t)}L(t)^†\ket{\psi^{\vb{e}_\mu}(η^\ast,t)} + \cc.
        \end{aligned}
      \end{equation}
    \end{description}

    This result allows us to calculate the energy flow in arbitrarily
    driven systems.
  \end{alertblock}
  \begin{block}{Possible Applications}
    \begin{itemize}
    \item Simulation of Thermal Quantum Machines
    \item Convergence Criteria: Energy Conservation, Calculating the
      same observable in multiple ways
    \item Quantification of Entanglement of System and Bath (Fisher
      Information of \(H_\inter\))
    \item \ldots
    \end{itemize}
  \end{block}
\end{column}

\separatorcolumn

\begin{column}{\colwidth}
  \begin{block}{Resources}
    \printbibliography
  \end{block}
\end{column}

\separatorcolumn
\end{columns}
\end{frame}

\end{document}
