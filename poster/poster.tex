\documentclass[draft]{beamer}

\pdfvariable suppressoptionalinfo 512\relax

% ====================
% Packages
% ====================

\usepackage[T1]{fontenc}
\usepackage{../hiromacros}

\usepackage{lmodern}
\usepackage[size=A0,orientation=portrait,scale=1.0]{beamerposter}
\usetheme{gemini}
\usecolortheme{mit}
\usepackage{graphicx}
\usepackage{booktabs}
\usepackage{tikz}
\usepackage{pgfplots}
\pgfplotsset{compat=1.14}
\usepackage{physics}
\usepackage{cleveref}
\usepackage[list=true, font=footnotesize, labelformat=brace]{subcaption}
\graphicspath{{figs/}}

% Bibliographic Stuff
\usepackage[backend=biber, language=english, style=phys]{biblatex}
\addbibresource{references.bib}

\DeclareDocumentCommand\vectorbold{ s m
}{\IfBooleanTF{#1}{\symbfit{#2}}{\symbf{#2}}} % Vector bold [star for
% Greek and italic Roman]

\makeatletter
\newcommand\thefontsize[1]{{#1 The current font size is: \f@size pt\par}}
\makeatother

% ====================
% Lengths
% ====================

% If you have N columns, choose \sepwidth and \colwidth such that
% (N+1)*\sepwidth + N*\colwidth = \paperwidth
\newlength{\sepwidth}
\newlength{\colwidth}
\setlength{\sepwidth}{0.066\paperwidth}
\setlength{\colwidth}{0.4\paperwidth}

\newcommand{\separatorcolumn}{\begin{column}{\sepwidth}\end{column}}

% ====================
% Title
% ====================

\title{Calculating Energy Flows in Strongly Coupled Open Quantum
  Systems with HOPS}

\author{\underline{Valentin Boettcher}\inst{1}, Richard Hartmann\inst{1},
  Konstantin Beyer\inst{1}, Walter Strunz\inst{1}}

\institute[shortinst]{\inst{1} Institute for Theoretical Physics, Dresden, Germany}

% ====================
% Footer (optional)
% ====================

\footercontent{
  \href{https://tu-dresden.de/mn/physik/itp/tqo/die-professur}{https://tu-dresden.de/mn/physik/itp/tqo/die-professur} \hfill
  Group for Theoretical Quantum Optics, TU-Dresden \hfill
  \href{mailto:valentin.boettcher@tu-dresden.de}{valentin.boettcher@tu-dresden.de}}
% (can be left out to remove footer)

% ====================
% Logo (optional)
% ====================

% use this to include logos on the left and/or right side of the header:
\logoleft{\includegraphics[height=3cm]{Logo_TU_Dresden.pdf}}

% ====================
% Body
% ====================

\begin{document}

\begin{frame}[t]
\begin{columns}[t]
\separatorcolumn

\begin{column}{\colwidth}
  \begin{block}{Premise}
    \begin{itemize}
    \item Application of thermodynamic notions to strongly coupled and
      non-Markovian quantum systems is non-trivial
      (\cite{Rivas2019Oct,Kato2016Dec,Strasberg2021Aug,Talkner2020Oct} and many
      more)
    \item Dynamics of bath and interaction hamiltonians plays an
      important role \(\rightarrow\) must not be neglected in the
      strong coupling regime
    \item The ``Hierarchy of Pure States''
      (HOPS~\cite{Hartmann2017Dec,Diosi1998Mar}) gives us the ability
      to simulate non-Markovian and strongly coupled open quantum
      systems exactly in a scalable way.
    \item Because HOPS simulates global dynamics \(\rightarrow\) gives
      access to certain bath dynamics \emph{with no additional effort}
    \end{itemize}
  \end{block}
  \begin{block}{NMQSD/HOPS}
    \begin{itemize}
    \item Consider the model of a general quantum system (\(H_\sys(t)\))
    coupled to \(N\) baths
    \begin{equation}
      \label{eq:generalmodel}
      H(t) = H_\sys(t) + ∑_{n=1}^N \qty[L_n^†(t)B_n + \hc] + ∑_{n=1}^NH_B\nth ,
    \end{equation}
    with \(B_n=∑_{λ} g_λ\nth a_λ\nth\) and
    \(H_B\nth=∑_λω_λ\nth \qty(b_λ\nth)^\dag b_λ\nth\)

    \item Leads to \emph{stochastic} Non-Markovian
    Quantum State Diffusion (NMQSD)
    \begin{equation}
      \label{eq:nmqsd}
      ∂_tψ_t(\vb{η}^\ast_t) = -\iu H(t) ψ_t(\vb{η}^\ast_t) +
      \vb{L}\cdot \vb{η}^\ast_tψ_t(\vb{η}^\ast_t) - ∑_{n=1}^N L(t)_n^†∫_0^t\dd{s}α_n(t-s)\fdv{ψ_t(\vb{η}^\ast_t)}{η^\ast_n(s)}
    \end{equation}
    \item Reduced state of the system is recovered through
      \(ρ=\mathcal{M}(ψ_t(\vb{η}^\ast_t)ψ_t^\dag(\vb{η}^\ast_t))\)

    \item With \(α_n(τ)=∑_{\mu}^{M_n}G_μ\nth\eu^{-W_μ\nth τ}\)
      we find for the Fock-space embedded hierarchy state \(\ket{Ψ}\)
    \begin{equation}
      \label{eq:fockhops}
      \begin{aligned}
        ∂_t\ket{Ψ} &= \qty[
                     \begin{aligned}
                       -\iu H_\sys + \vb{L}\cdot\vb{η}^\ast &-
                                                              ∑_{n=1}^N∑_{μ=1}^{M_n}b_{n,μ}^\dag b_{n,μ} W\nth_μ \\
                                                            &\qquad+
                                                              \iu ∑_{n=1}^N∑_{μ=1}^{M_n} \sqrt{G_{n,μ}} \qty(b^†_{n,μ}L_n +
                                                              b_{n,μ}L^†_n)
                     \end{aligned}
                     ] \ket{Ψ}.
      \end{aligned}
    \end{equation}
    \item Truncating hierarchy depth \(\kmat\) in \cref{eq:fockhops}
      yields numeric method

    \item Finite temperature \(\rightarrow\) substitute
    \(B(t)\rightarrow B(t)+ξ(t)\), \(ξ\) being a suitable
    Gaussian process
    \item \(\exists\) nonlinear method which improves convergence drastically
    \end{itemize}

    See~\cite{Hartmann2017Dec} for details.
  \end{block}

  \begin{alertblock}{Bath Observables}
    \begin{itemize}
    \item \Cref{eq:nmqsd,eq:fockhops} \(\implies\) correspondence
      \(B^n(t) \leftrightarrow ψ^\kmat\) {\tiny(\(\abs{\kmat}=n\))}

    \item Can calulate observables of type
    \(O_\sys\otimes (B^a)^\dag B^b\) + time derivatives
    \end{itemize}


    We consider the zero temperature case with one bath in the linear
    method.

    \begin{description}
    \item[Bath Energy Flow]
      \begin{equation}
        \label{eq:heatflowdef}
        \begin{aligned}
          J &= - \dv{\ev{H_\bath}}{t}  = \ev{L(t)^†∂_t B(t) + L(t)∂_t
              B^†(t)}_\inter \\
            &=-\i \mathcal{M}_{η^\ast}\bra{\psi(η,
              t)}L(t)^†\dot{D}_t\ket{\psi(η^\ast,t)} + \cc\\
            &= - ∑_\mu\sqrt{G_\mu}W_\mu
              \mathcal{M}_{η^\ast}\bra{\psi^{(0)}(η,t)}L(t)^†\ket{\psi^{\vb{e}_\mu}(η^\ast,t)} + \cc
        \end{aligned}
      \end{equation}
      \begin{itemize}
      \item Expectation value of bath energy flow \(\leftrightarrow\)
        first level hierarchy states in a transparent and easy to calculate
        manner
      \end{itemize}
    \item[Interaction Energy]
      \begin{itemize}
      \item Similar expression exists the expectation value of the
        interaction energy
      \end{itemize}


      \begin{equation}
        \label{eq:intexp}
        \begin{aligned}
          \ev{H_\inter} &=-\i \mathcal{M}_{η^\ast}\bra{\psi(η,
          t)}L(t)^†D_t\ket{\psi(η^\ast,t)} + \cc \\
                        &=  ∑_\mu\sqrt{G_\mu}
                          \mathcal{M}_{η^\ast}\bra{\psi^{(0)}(η,t)}L(t)^†\ket{\psi^{\vb{e}_\mu}(η^\ast,t)} + \cc.
        \end{aligned}
      \end{equation}
    \end{description}

    This result allows us to calculate the energy flow in
    \textbf{arbitrarily driven systems} for a \textbf{wide temperature
      range} and with \textbf{arbitrary BCF}.
  \end{alertblock}

  \begin{block}{Comparison to an Analytic Solution}
    \begin{columns}
      \begin{column}{.5\colwidth}
        \begin{itemize}
        \item Quantum Brownian Motion like Model
          \begin{equation}
            \label{eq:hamiltonian}
            \begin{aligned}
              H = ∑_{i\in\qty{1,2}} &\qty[H^{(i)}_O + q_iB^{(i)} +
                                      H_B^{(i)}]\\
                                    &\quad+ \frac{γ}{4}(q_1-q_2)^2
            \end{aligned}
          \end{equation}
          where \(H_O^{(i)}= \frac{Ω_i}{4}\qty(p_i^2+q_i^2)\)
        \item Exact solution via exponential expansion of the BCF in the
          Heisenberg Picture \(\rightarrow\) easy access to bath energy
          flow
        \item Here \tval{analytic/omega}, \tval{analytic/gamma},
          \(α(τ)=η (1+\iu ω_cτ)^{-(s+1)}\) with
          \tval{analytic/cutoff_freq}, \tval{analytic/bcf_zero}
        \end{itemize}
      \end{column}
      \begin{column}{.5\colwidth}
        \begin{figure}[H]
          \centering
          \plot{analytic/flow}
          \caption{\label{fig:brownian}The bath energy flows \cref{eq:heatflowdef} for the
            hot (lower line) and the cold (upper line) bath. The
            analytical and numerical results are compatible.
            (\tval{analytic/samples} samples)}
        \end{figure}
      \end{column}
    \end{columns}
  \end{block}


  \begin{block}{Possible Applications}
    \begin{itemize}
    \item Simulation of Thermal Quantum Machines
    \item Convergence Criteria: Energy Conservation, Calculating the
      same observable in multiple ways
    \item Quantification of Entanglement of System and Bath (Fisher
      Information of \(H_\inter\))
    \item \ldots
    \end{itemize}
  \end{block}
\end{column}

\separatorcolumn

\begin{column}{\colwidth}
  \begin{block}{Simple Toy Model and BCF Dependence}
    \begin{itemize}
    \item Spin-boson like Model coupled to a zero temperature bath
      \begin{equation}
        \label{eq:spinbos}
        H_\sys= \frac12 σ_z,\, L=\frac12 σ_x,\, α(τ)=η (1+\iu ω_cτ)^{-(s+1)}
      \end{equation}
    \item Memory time \(\sim 1/ω_c\) has qualitative influence on the
      bath energy flow
    \end{itemize}
    \begin{figure}[H]
      \centering
      \begin{subfigure}[t]{.49\columnwidth}
        \plot{one_bath/omega_interaction}
        \caption{\label{fig:omega_ints}\tval{one_bath/omega_bcf_str}}
      \end{subfigure}
      \begin{subfigure}[t]{.49\columnwidth}
        \plot{one_bath/delta_interaction}
        \caption{\tval{one_bath/delta_bcf_wc}}
      \end{subfigure}
      \caption{The interaction energy expectation value for different
        cutoff frequencies and coupling strengths. The dashed lines
        are obtained using energy conservation while the solid lines
        are the result of direct calculation. The percentages in the
        legend tell how many points are compatible within one standard
        deviation. The Statistical error estimate is smaller than the
        line width. \(N=5\cdot 10^5\) trajectories have been used.}
    \end{figure}
  \end{block}
  \begin{block}{Initial Slip}
    \begin{itemize}
    \item For \emph{very} short times \(\rightarrow\) \(H_\sys\approx
      0\), Origin of the \emph{``Initial Slip''} spike in
      \cref{fig:brownian}:
      \begin{equation}
        \label{eq:purede}
        \ev{\dot{H}_\bath } = -2 ∫_0^t\dd{s}\ev{L(t)L(s)} \Im[\dot{α}(t-s)].
      \end{equation}
    \item Determines ultra short-time shape of \emph{all} trajectories
    \end{itemize}
    \begin{figure}[H]
      \centering
      \begin{subfigure}[t]{.49\linewidth}
        \plot{one_bath/initial_slip}
        \caption{\label{fig:initslipconst}The bath energy flows for the same settings as in
          \cref{fig:omega_ints}. The dashed lines correspond to \cref{eq:purede}.}
      \end{subfigure}
      \begin{subfigure}[t]{.49\linewidth}
        \plot{modcoup/initial_slip_modcoup}
        \caption{Same as \cref{fig:initslipconst}, but for modulated
          coupling (``Smoothstep'' functions, see inset).}
      \end{subfigure}
    \end{figure}
  \end{block}
  \begin{block}{Modulating the Coupling}
    \begin{itemize}
    \item Same model as above \cref{eq:spinbos}, but with \(L(τ) =
      \sin^2(\frac{Δ}{2} τ)σ_x\)
    \item Question: How much energy can be extracted from a system
      connected to a single bath? (Ergotropy)
      \begin{itemize}
      \item Answer: less than
        \(ΔE_{\mathrm{max}}=\frac{1}{β}\qrelent{ρ}{ρ_β}\)
      \end{itemize}
    \end{itemize}
    \begin{figure}[H]
      \centering
      \begin{subfigure}[t]{.49\linewidth}
        \plot{modcoup/omegas_total}
        \caption{\label{fig:omega_total}The total energy for
          \tval{modcoup/omega_delta} and \tval{modcoup/omega_alpha}
          but varying cutoff. Energy is normalized to the
          ergotropy. The dashed vertical lines show where \(α(τ) = α(0)/300\).}
      \end{subfigure}
      \begin{subfigure}[t]{.49\linewidth}
        \plot{modcoup/flow_interaction_overview}
        \caption{Same situation as in \cref{fig:omega_total}. All
          quantities are normalized to the \(ω_c=1\)
          case.}
      \end{subfigure}
      \begin{subfigure}[t]{.49\linewidth}
        \plot{modcoup/flow_interaction_overview}
        \caption{Maximum power for \(10\) periods and different
          modulations \(Δ\).}
      \end{subfigure}
      \begin{subfigure}[t]{.49\linewidth}
        \plot{modcoup/flow_interaction_overview}
        \caption{Maximum power for \(10\) periods and different
          coupling strengths \(α(0)\).}
      \end{subfigure}
    \end{figure}
  \end{block}
  \begin{block}{Resources}
    {\AtNextBibliography{\tiny} \printbibliography}
  \end{block}
\end{column}

\separatorcolumn
\end{columns}
\end{frame}

\end{document}
