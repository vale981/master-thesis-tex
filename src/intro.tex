\chapter{Introduction and Preliminaries}
\label{chap:intro}

The exact treatment of strongly coupled and non Markovian open quantum
systems is a challenging problem.  If no analytical solution is
available, numerical methods have to be relied upon. Notably there are
HEOM\fixme{cite more}.

Besides the reduced dynamics of the system, the field of quantum
thermodynamics has attracted much interest. Quantum thermodynamics is,
among other issues, concerned with extending the standard
phenomenological thermodynamic notions to microscopic systems coupled
to macroscopic baths. This setting may make it possible to make
rigorous microscopic definitions of thermodynamic quantities such as
internal energy, heat and work that are consistent with the laws of
thermodynamics. There is no consensus on this matter yet, as is
demonstrated by the plethora of proposals and discussions in
\cite{Rivas2019Oct,Talkner2020Oct,Motz2018Nov,Wiedmann2020Mar,Senior2020Feb,Kato2015Aug,Kato2016Dec,Strasberg2021Aug,Talkner2016Aug,Bera2021Feb,Bera2021Jun,Esposito2015Dec}.

As system and baths may not be regarded as completely separable
entities in a strong coupling
regime~\cite{Rivas2019Oct,Esposito2015Dec}, an insight into the
dynamics of the bath is crucial. In some settings
\cite{Kato2016Dec,Lobejko2021Feb, Strasberg2021Aug}, such as cyclic
heat engines, the change in the bath energies is a quite suitable
definition of heat, as is expounded in
\cref{sec:basic_thermo,sec:operational_thermo}.

As it turns out, the framework of the ``Non Markovian State
Diffusion'' (NMQSD)~\cite{Diosi1998Mar}, which will be reviewed
in~\cref{sec:quick_hops}, allows access to certain bath related
observables such as the time derivative of the bath energy expectation
value and the interaction Hamiltonian expectation value. The

In \cref{sec:basic_thermo}, some operational quantum thermodynamical
questions will be discussed and some light will be shed on the
necessity of infinite baths.

\Cref{chap:flow} presents the main result of this thesis, namely
formulas to calculate expectation values of bath related
observables. Finally, we will discuss some numerical applications of
the result in \cref{chap:numres}. These application include a
\fixme{references}.

In \cref{chap:hops_notes}, some HOPS related subjects are discussed,
including a derivation of the HOPS for multiple baths embedded in an
auxiliary bosonic Fock space in \cref{sec:multihops}.

\newpage

\section{Open Systems, the NMQSD and HOPS}
\label{sec:quick_hops}

The basic and most general model which forms the foundation of all
matters discussed in this is a general quantum system \(H_\sys(t)\)
coupled to \(N\) baths of harmonic oscillators
\begin{equation}
  \label{eq:generalmodel}
  H(t) = H_\sys(t) + ∑_{n=1}^N \qty[L_n^†(t)B_n + \hc] + ∑_{n=1}^NH_B\nth ,
\end{equation}
with \(B_n=∑_{λ} g_λ\nth a_λ\nth\) and
\(H_B\nth=∑_λω_λ\nth \qty(a_λ\nth)^\dag a_λ\nth\). The \(a_λ\) are
bosonic annihilation operators and the \(L_n\) are arbitrary not
necessarily hermitian operators system Hilbert space. Sometimes
\cref{eq:generalmodel} is called the ``Standard Model of Open
Systems''. Throughout the work we set \(\hbar=c=1\).

Despite the simple structure of the baths, \cref{eq:generalmodel} is
generally very hard to solve beyond weak coupling strengths and the
secular approximation~\cite{Rivas2012}. The ``Non Markovian Quantum
State Diffusion'' (NMQSD)~\cite{Diosi1998Mar} approach allows to
recast \cref{eq:generalmodel} into a stochastic differential equation
in which the bath degrees of freedom are accounted for by Gaussian
stochastic processes. This drastic reduction of the number of degrees
of freedom also leads to an efficient numerical method, the
``Hierarchy of Pure States'' (HOPS).

The basics of the NMQSD will be briefly reviewed in
\cref{sec:nmqsd_basics} as will the basics of HOPS in
\cref{sec:hops_basics}. A more detailed account may be found in
\cref{sec:multihops} as well as \cite{RichardDiss}.

\subsection{Non Markovian Quantum State Diffusion}
\label{sec:nmqsd_basics}

We begin by considering a single zero temperature bath in the ground
state \(\ket{0}\). The total system-bath state may then be expanded in
a Bargmann coherent state basis~\cite{klauder1968fundamentals} with
respect to the bath degrees of freedom
\begin{equation}
  \label{eq:projected_single}
  \ket{ψ(t)} = ∫{\frac{\dd{\vb{z}}}{π^{N}}\eu^{-\abs{\vb{z}}^2}}\ket{ψ(t,\vb{z})^\ast}\ket{\vb{z}},
\end{equation}
where \(\vb{z}\) is a vector of coherent state labels \(z_λ\) for each
environment oscillator.

After transforming \cref{eq:generalmodel} into the interaction picture
with respect to \(H_B\) and using the properties of the coherent
states (\(\mel{z_λ}{a_λ}{ψ}\rightarrow ∂_{z_λ^\ast}\braket{z_λ}{ψ}\),
\(\mel{z_λ}{a_λ^\dag}{ψ}\rightarrow z_λ^\ast\braket{z_λ}{ψ}\)) we
arrive at
\begin{equation}
  \label{eq:nmqsd_single}
  ∂_tψ_t(η^\ast_t) = -\iu H ψ_t(η^\ast_t) +
  L {η}^\ast_tψ_t({η}^\ast_t) - L^†∫_0^t\dd{s}α(t-s)\fdv{ψ_t({η}^\ast_t)}{η^\ast_s},
\end{equation}
where \(α\) is the zero temperature bath correlation function (BCF)
\begin{equation}
  \label{eq:bcfdef}
  α(t-s) = \ev{B(t)B(s)} = ∑_λ \abs{g_λ}^2\,\eu^{-\iu ω_λ (t-s)}
\end{equation}
and \(η_t\) is a Gaussian stochastic process obeying
\begin{equation}
  \label{eq:single_processescorr}
  \begin{aligned}
      \mathcal{M}(η^\ast_t) &=0, & \mathcal{M}(η_tη_s) &= 0,
      & \mathcal{M}(η_tη_s^\ast) &= α(t-s).
  \end{aligned}
\end{equation}

Note that the BCF \(α\) is usually defined as Fourier transform of the
spectral density
\begin{equation}
  \label{eq:specdens}
  J(ω) = {π} ∑_λ \abs{g_λ}^2 δ(ω-ω_λ).
\end{equation}
One then usually performs a continuum limit so that \(J(ω)\) becomes
``smeared out'' to a smooth function and \(α(τ)\) decays to zero for
\(τ\rightarrow ∞\). This behavior leads to \cref{eq:nmqsd_single}



\subsection{Hierarchy of Pure States}
\label{sec:hops_basics}
