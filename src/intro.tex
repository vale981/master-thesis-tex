\chapter{Introduction}
\label{chap:intro}
The field of quantum thermodynamics has attracted much
interest~\cite{Talkner2020Oct,Rivas2019Oct,Riechers2021Apr,Vinjanampathy2016Oct,Binder2018,Kurizki2021Dec}. Quantum
thermodynamics is, among other issues, concerned with extending the
standard phenomenological thermodynamic notions to microscopic systems
coupled to macroscopic baths. This setting may make it possible to
formulate rigorous microscopic definitions of thermodynamic quantities
such as internal energy, heat and work that are consistent with the
well-known laws of thermodynamics. There is no consensus on this
matter yet, as is demonstrated by the plethora of proposals and
discussions in
\cite{Rivas2019Oct,Talkner2020Oct,Motz2018Nov,Wiedmann2020Mar,Senior2020Feb,Kato2015Aug,Kato2016Dec,Strasberg2021Aug,Talkner2016Aug,Bera2021Feb,Bera2021Jun,Esposito2015Dec}.

The verification and evaluation of these issues in nontrivial
systems calls for an additional insight into the exact dynamics of
open quantum systems.  As system and baths may not be regarded as
separable entities in a strong coupling
regime~\cite{Rivas2019Oct,Esposito2015Dec} due to entanglement and
finite interactions strengths, access to the dynamics of system,
interaction and bath is required. This constitutes the fundamental
motivation for this work.

The exact treatment of strongly coupled and non Markovian open quantum
systems is a challenging problem.  If no analytical solution is
available, numerical methods have to be relied upon. Notably there are
pertubative methods such as the Redfield equations
\cite{Davidovic2020Sep} and also exact methods like the Hierarchical
Equations of Motion \emph{HEOM}~\cite{Tanimura1990Jun,Tang2015Dec},
multilayer MCTDH~\cite{Wang2010May}, TEMPO~\cite{Strathearn2018Aug}
and the Hierarchy of Pure States
\emph{HOPS}~\cite{Suess2014Oct}\footnote{See \cite{RichardDiss} for a
  detailed account.}. Although the focus of these methods is on the
reduced system dynamics, exact treatments of open systems can provide
access to the global unitary evolution of the system and the baths.

% In some settings
% \cite{Kato2016Dec,Lobejko2021Feb, Strasberg2021Aug}, such as cyclic
% heat engines, the change in the bath energies is a quite suitable
% definition of heat, as is expounded in
% \cref{sec:basic_thermo,sec:operational_thermo}.

In this work we will focus on the framework of the ``Non Markovian
State Diffusion'' (\emph{NMQSD})~\cite{Diosi1998Mar}, which will be
reviewed in~\cref{sec:open_systems,sec:nmqsd_basics}. We will show in
\cref{chap:flow} that the NMQSD allows access to interaction and bath
related quantities. This novel application of the formalism
constitutes the main result of this work.

Based on the NMQSD and inspired by the ideas behind HEOM a numerical
method, the Hierarchy of Pure States
\emph{HOPS}~\cite{RichardDiss,Hartmann2017Dec}, can be formulated and
will be briefly reviewed in \cref{sec:hops_basics}.

The results of \cref{chap:flow}, most importantly the calculation of
bath and interaction energy expectation values, can be easily
implemented within this numerical framework. By doing so we will
elucidate the role of certain features inherent to the method. The
most general case we will be able to handle is a system coupled to
multiple baths of differing temperatures under arbitrary
modulation. As HOPS on its own is already a method with a very broad
range of applicability~\cite{RichardDiss}, we will find it to be
suitable for the exploration of thermodynamical settings.

\Cref{chap:analytsol} is concerned with the analytical solution of the
well known model for Quantum Brownian Motion (QBM). This solution will
be applied to exactly calculate bath related quantities to provide a
benchmark for HOPS and the results of \cref{chap:flow}.

The implementation and systematic application of this benchmark is
performed in \cref{chap:numres}. Further, the spin boson model will be
explored as an example of a system that is not exactly solvable in
general.

Finally, we will turn to some thermodynamical questions in
\cref{sec:therm_results} both from a conceptual angle in
\cref{sec:basic_thermo} and through the exploration of concrete
systems in \cref{sec:singlemod,sec:otto}.

The aim of this work is to demonstrate the validity of the HOPS
framework and its suitability for thermodynamical applications. In
this spirit spin-boson like models are explored to provide an overview
of the capabilities of the method and ideas for future
studies. Although those are mentioned throughout the work, some
promising avenues are being highlighted in
\cref{cha:concl-ideas-future}.

\section{Open Quantum Systems}
\label{sec:open_systems}
Quantum physics' most important equation, the Schr\"odinger equation,
allows us to predict the future of a system knowing its initial state
if restrict ourself to pure states. Writing it down
\begin{equation}
  \label{eq:schroedinger}
  \iu ∂_{t} \ket{ψ(t)} = H \ket{ψ(t)},
\end{equation}
we find that we need only to specify a \emph{Hamiltonian} \(H\) that
acts on our system state which is an element of a Hilbert space of
some possible infinite dimension \(N\). Throughout the work we set
\(\hbar=c=1\).

We call the time evolution generated by \cref{eq:schroedinger}
\emph{Unitary}, as it preserves the norm of a state and is reversible.
Given any time independent Hamiltonian we may write down the time
evolution operator to solve the Schr\"odinger equation
\begin{equation}
  \label{eq:time_evo_op}
  U(t, t_{0})=\eu^{-\iu H (t-t_{0})},\; U(t, t_{0})^\dag U(t, t_{0}) =
  \id,\; U(t, t_{0})\ket{ψ(t_{0})} = \ket{ψ(t)}.
\end{equation}

For time independent Hamiltonians the Schr\"odinger equation describes
a closed system which constitutes, within the scope of the problem in
question, the whole universe. In general, it is very hard to find a
closed expression for \cref{eq:time_evo_op}, except for very special
cases. Either we recourse to approximations or we apply numerical
methods to solve \cref{eq:schroedinger}.

When the Hilbert space dimension is small, its numerical solution is
straight forward. But in more realistic scenarios we may still be
interested in a small system, but we cannot neglect the interaction of
that system with a much larger environment sometimes consisting of
infinite degrees of freedom. If the atmosphere of the earth would be
neglected when describing the descent of a space reentry capsule we
would arrive at fatally wrong results. Similarly, modern applications
of quantum physics deal with systems that undergo quantum evolution
under conditions that are not consistent with an isolated
system.

Specifically in quantum computing~\cite{Gill2022Jan}, the effect of
environmental interactions poses a major problem.  Also, in quantum
thermodynamics, we are often concerned with irreversible dynamics that
converge to a steady state or a limit cycle. This is only possible, if
the bath possesses infininite degrees of freedom \cite{Breuer2002Jun}.

As a classical example, Stokes drag models the influence of a viscous
fluid on spherical objects and can be implemented by adding a velocity
dependent term to the equation of motion of our object,
\begin{equation}
  \label{eq:newton}
  \ddot{x} = F - α \dot{x}.
\end{equation}
We still retain all information about the system, the particle, having
accounted effectively for an environment, the fluid by introducing a
term that breaks energy conservation.

In quantum physics, we find that the situation is more complicated.
Writing down a Hamiltonian we have to account for both system and
environment in a composite Hilbert space
\(\hilb=\hilb_{\sys}\otimes\hilb_{\bath}\)
\begin{equation}
  \label{eq:general_open}
  H = H_{\sys} \otimes \id_{\bath} + \id_{\sys} \otimes H_{\bath} + H_{\inter},
\end{equation}
where \(\sys\) marks the system, \(\bath\) marks the environment (or
bath) and \(H_{\inter}\) models their interaction.

Although the global state of system and environment may be pure,
entanglement of system and environment leads to the effect, that we
may know the system state only as a statistical mixture of states
called the \emph{reduced state}. No part of a composite system may be
in general be known as ``precicely'' as the whole. This is a major
departure from classical physics.

Starting from a possibly mixed global state \(ρ(t)\) we find, that to
compute the dynamics of all observables \(O_{\sys}\) which only act on
the system Hilbert space it is sufficient to know
\(ρ_{\sys}(t)=\tr_{\bath}[ρ(t)]\), the reduced system state.

The partial trace \(\tr_{\bath}\) averages over all bath degrees of
freedom and removes them from explicit consideration. This is a very
useful device, as the environment usually has a Hilbert space
dimension that is much too large for practical
calculations. Especially in numerics this fact is important, as even
an environment consisting of \(40\) two level systems would consume
\(16\) tebibytes of memory when stored as double precision complex
floating point numbers.

We will discover in \cref{chap:flow} that although it is impractical
to work with the bath state directly, it is still possible gain access
to bath-global observables such as the change in the expectation value
of the bath energy.

Under certain assumptions, most importantly that of weak coupling
\(\ev{H_{\inter}}\approx 0\), a pertubative treatment of the
environment yields an evolution equation, called a \emph{master
  equation}, that only contains the system state
\(ρ_{\sys}\)~\cite[p. 115 ff.]{Breuer2002Jun,Rivas2012}, leading to
irreversible non unitary dynamics.

A special master equation often called
Gorini–Kossakowski–Sudarshan–Lindblad equation, or GKSL equation in
short, and has the form
\begin{equation}
  \label{eq:gksl}
  \dot{ρ}_{\sys} = -\iu \comm{H}{ρ_{\sys}} + \mathcal{D}[ρ_{\sys}] = \mathcal{L}[ρ_{\sys}],
\end{equation}
where \(\mathcal{D}\) is called the \emph{dissipator} which adds
non-unitary dynamics to the von Neumann equation and \(H\) is a
unitary contribution not necessarily equal to \(H_{\sys}\).

Integrating \cref{eq:gksl} leads to a map
\(ρ_{\sys}(t) = \mathcal{E}_{t}(ρ_{\sys}(0))\).  The evolution
generated by \cref{eq:gksl} is called Markovian, as
\(\mathcal{E}_{t+s}= \mathcal{E}_{t}\circ\mathcal{E}_{s}\). More
fundamentally, this is due to the fact that at some point one assumes,
that the bath has no ``memory''\footnote{This does not mean that the
  global state has always the form
  \(ρ_{\sys}(t)\otimes ρ_{\bath}(t)\)~\cite{Rivas2012}.}. Without
getting into details, one may say that the characteristic time scales
upon which correlation functions of bath observables decay should be
much smaller than the time scales of the system.

If one endeavors to drop the assumptions of weak coupling and of
Markovian dynamics, the situation becomes more complicated. But when
introducing a concrete model of the bath we find that
\cref{eq:schroedinger} can be recast into a form that allows for an
exact numerical solution. As alluded to above, the great advantage of
this exact approach is that although we solve for the reduced state
\(ρ_{\sys}\), we essentially compute the unitary dynamics of
system. Thus, we can additionally retain some information about the
bath which allows us to quantify the change in expected bath energy
and also the expectation value of the interaction Hamiltonian.

\Cref{sec:nmqsd_basics} will introduce the general model whose
solution will be made feasible with the introduction of the \emph{Non
  Markovian Quantum State Diffusion} (NMQSD). The numerical
implementation of the NMQSD, the \emph{Hierarch of Pure States}
(HOPS), will be the topic of \cref{sec:hops_basics}.

A more detailed account of both subjects can be found in
\cref{sec:multihops} as well as \cite{RichardDiss}.


\section{The Non Markovian Quantum State Diffusion (NMQSD)}
\label{sec:nmqsd_basics}

We will now introduce the most general form of the models that will be
discussed in this thesis. This model has a wide applicability and many
microscopic systems can be cast into its form
\cite{Strunz2001Habil}\cite[chap. 2]{RichardDiss}, although there
certainly exist limits of applicability~\cite{Caldeira2014Mar}.

Consider a general quantum system \(H_\sys(t)\) coupled to \(N\) baths
of harmonic oscillators\footnote{For instance, the electromagnetic field.}
\begin{equation}
  \label{eq:generalmodel}
  H(t) = H_\sys(t) + ∑_{n=1}^N \qty[L_n^†(t)B_n + \hc] + ∑_{n=1}^NH_B\nth ,
\end{equation}
with \(B_n=∑_{λ} g_λ\nth a_λ\nth\) and
\(H_B\nth=∑_λω_λ\nth \qty(a_λ\nth)^\dag a_λ\nth\). The \(a_λ\nth\) are
bosonic annihilation operators acting on \(n\)th bath Hilbert space
and the \(L_n\) are arbitrary not necessarily hermitian operators
acting on the system Hilbert space. Quite general baths can be mapped
to this model in the thermodynamic limit, when the bath consists of an
infinite number of independent systems as was shown in
\cite{Makri1999Apr}.

Despite the simple structure of the baths, \cref{eq:generalmodel} is
generally very hard to solve beyond weak coupling strengths as has
been detailed in~\cref{sec:open_systems}. The ``Non Markovian Quantum
State Diffusion'' \cite{Diosi1998Mar} approach allows to
recast \cref{eq:generalmodel} into a stochastic differential equation
in which the bath degrees of freedom are accounted for by Gaussian
stochastic processes and a memory term.

Here we only consider a single zero temperature bath initially in the
ground state \(\ket{0}\). For more complete and general account see
\cite{RichardDiss,Strunz2001Habil,Diosi1998Mar,Hartmann2017Dec,Hartmann2021Aug}
and \cref{sec:hops_multibath}.

The total system-bath state may then be expanded in a Bargmann
unnormalized coherent state basis~\cite{klauder1968fundamentals} with
respect to the bath degrees of freedom
\begin{equation}
  \label{eq:projected_single}
  \ket{ψ(t)} = ∫{\frac{\dd{\vb{z}}}{π^{N}}\eu^{-\abs{\vb{z}}^2}}\ket{ψ(t,\vb{z}^\ast)}\ket{\vb{z}},
\end{equation}
where \(\vb{z}\) is a vector of coherent state labels \(z_λ\) for each
environment oscillator. The reason for using unnormalized coherent
states is that this leads to \(\ket{ψ(t,\vb{z}^{\ast})}\) being a
holomorphic function of \(\vb{z}^\ast\).


After transforming \cref{eq:generalmodel} into the interaction picture
with respect to \(H_\bath\) and using the properties of the coherent
states (\(\mel{z_λ}{a_λ}{ψ}\rightarrow ∂_{z_λ^\ast}\braket{z_λ}{ψ}\),
\(\mel{z_λ}{a_λ^\dag}{ψ}\rightarrow z_λ^\ast\braket{z_λ}{ψ}\)) we
arrive at an equation for \(\ket{ψ(t,\vb{z}^{\ast})}\)
\begin{equation}
  \label{eq:nmqsd_single_proto}
  ∂_t\ket{ψ(t,\vb{z}^{\ast})} = -\iu H \ket{ψ(t,\vb{z}^{\ast})} - \iu
  L ∑_{λ}g_{λ}^\ast z_{λ}^\ast \eu^{\iu ω_{λ} t}
  \ket{ψ(t,\vb{z}^{\ast})} - \iu L^\dag ∑_{λ} g_{λ}\eu^{-\iu ω_{λ} t}\pdv{}{z_{λ}^\ast}\ket{ψ(t,\vb{z}^{\ast})}.
\end{equation}

From this point on there are multiple avenues open to us. We choose
the canonical one of \cite{Strunz2001Habil}, but there is also a
time-discrete derivation that avoids functional derivatives
in~\cite{Hartmann2021Aug}.

We shift the perspective and define~\cite{RichardDiss,Strunz2001Habil}
\begin{equation}
  \label{eq:single_process}
  η^\ast_{t} = -\iu ∑_λg_λ^{\ast} z_λ^{\ast}\eu^{\iu ω_λ t}.
\end{equation}
Using
\(\pdv{z_λ^{\ast}}=∫\dd{s}\pdv{η^\ast_{s}}{z_λ^{\ast}}\fdv{η^\ast_{s}}\)
and interpreting \(\ket{ψ(t,\vb{z}^{\ast})}\) as a functional of
\(η_{t}^\ast\) we arrive at
\begin{equation}
  \label{eq:nmqsd_single}\tag{NMQSD}
  ∂_t\ket{ψ(η^\ast_t, t)} = -\iu H \ket{ψ(η^\ast_t, t)} +
  L {η}^\ast_t\ket{ψ({η}^\ast_t, t)} -
  L^†∫_0^t\dd{s}α(t-s)\fdv{\ket{ψ({η}^\ast_t, t)}}{η^\ast_s},
\end{equation}
where \(α\) is the zero temperature bath correlation function (BCF)
\begin{equation}
  \label{eq:bcfdef}
  α(t-s) = ∑_λ \abs{g_λ}^2\eu^{-\iu ω_λ (t-s)}
\end{equation}
and \(η_t\) is a Gaussian stochastic process obeying
\begin{equation}
  \label{eq:single_processescorr}
  \begin{aligned}
      \mathcal{M}(η_t) &=0, & \mathcal{M}(η_tη_s) &= 0,
      & \mathcal{M}(η_tη_s^\ast) &= α(t-s).
  \end{aligned}
\end{equation}
The functional \(\ket{ψ(η_{t}^\ast)}\) is essentially the state of the
system relative to some coherent bath state, encoded in the value of
the stochastic process.

The statistics of the process follow from interpreting
\cref{eq:projected_single} in a Monte-Carlo sense and thus the reduced
system state may then be recovered by averaging over all trajectories
\begin{equation}
  \label{eq:recover_rho}
  ρ_{\sys}(t) = \tr_{\bath}\bqty{\ketbra{ψ(t)}} =
 ∫{\frac{\dd{\vb{z}}}{π^{N}}\eu^{-\abs{\vb{z}}^2}}\ketbra{ψ(t,\vb{z})}{ψ(t,\vb{z}^\ast)}=
  \mathcal{M}_{η_{t}^\ast}\bqty{\ketbra{ψ(η_t, t)}{ψ(η^\ast_t, t)}}.
\end{equation}

Note that the BCF \(α\) relates to the so called spectral density
\begin{equation}
  \label{eq:specdens}
  J(ω) = {π} ∑_λ \abs{g_λ}^2 δ(ω-ω_λ)
\end{equation}
via the Fourier transform.

One then usually performs a continuum limit so that \(J(ω)\) becomes
``smeared out'' to a smooth function and \(α(τ)\) decays to zero for
\(τ\rightarrow ∞\).

The NMQSD formalism shows that we can treat an infinite environment
with a stochastic differential equation in which only objects of
system dimension appear. Note also, that we can allow an explicit time
dependence of \(L\) and \(H\) without alterations to
the NMQSD.

Equation \cref{eq:nmqsd_single} does not preserve the norm of the
state, leading to suboptimal convergence of \cref{eq:recover_rho}.
When recovering the system state, we would like to average over
normalized states
\begin{equation}
  \label{eq:norm_av}
  ρ_{\sys}(t) =
 ∫{\frac{\dd{\vb{z}}}{π^{N}}\eu^{-\abs{\vb{z}}^2}}
 \braket{ψ(t,\vb{z})}{ψ(t,\vb{z}^\ast)} \frac{\ketbra{ψ(t,\vb{z})}{ψ(t,\vb{z}^\ast)}}{\braket{ψ(t,\vb{z})}{ψ(t,\vb{z}^\ast)}}.
\end{equation}

This can be achieved~\cite{RichardDiss,Suess2014Oct} by using a
co-moving shifted stochastic process
\begin{equation}
  \label{eq:shifted_proc}
  \tilde{η}_{t}^\ast= η^\ast_{t} + ∫_{0}^{t}\dd{s} α^\ast(t-s) \ev{L^\ast}_{s},
\end{equation}
where
\(\ev{L^\dag}_{t}=\mel{ψ(\tilde{η}_{t}, t)}{L^\dag}{
ψ(\tilde{η}_{t}^\ast,t)}\). The origin of this shift lies in the
study of the Husimi \(Q\) function~\cite{Cartwright1976Jan} of the
bath
\(Q_{t}(\vb{z}, \vb{z}^\ast) = \eu^{-\abs{z}^{2}} π^{-N}
\braket{ψ(t,\vb{z})}{ψ(t,\vb{z}^\ast)}\). Shifting the process, or on
a deeper level the stochastic state labels, amounts to importance
sampling for each time step.

This leads to the nonlinear NMQSD equation~\cite{Diosi1998Mar}
\begin{equation}
  \label{eq:nmqsd_nonlin_single}
  ∂_t\ket{ψ(\tilde{η}^\ast_t, t)} = -\iu H \ket{ψ(\tilde{η}^\ast_t, t)} +
  L {\tilde{η}}^\ast_t\ket{ψ(\tilde{η}^\ast_t, t)} - \pqty{L^†
    -\ev{L^\dag}_{t}}∫_0^t\dd{s}α(t-s)\fdv{\ket{ψ({\tilde{η}}^\ast_t, t)}}{\tilde{η}^\ast_s}.
\end{equation}
There is a subtlety concerning the functional derivative that won't be
discussed here but in \cite{Hartmann2021Aug,RichardDiss} or
\cref{sec:nonlin_flow}.  Crucially, the system state is now recovered
through
\begin{equation}
  \label{eq:recover_rho_nonlinear}
  ρ_{\sys}(t) =
  \mathcal{M}_{\tilde{η}_{t}^\ast}\bqty{\frac{\ketbra{ψ(\tilde{η}_t, t)}{ψ(\tilde{η}^\ast_t,t)}}{\braket{ψ(\tilde{η}_t, t)}{ψ(\tilde{η}^\ast_t,t)}}},
\end{equation}
so that all trajectories contribute with ``equal weight''.

\section{The Hierarchy of Pure States (HOPS)}
\label{sec:hops_basics}
The equation \cref{eq:nmqsd_single} has removed the bath degrees of
freedom from explicit consideration, replacing them with a Gaussian
stochastic process and a rather complicated term containing a memory
integral and a functional derivative
\begin{equation}
  \label{eq:complicated_term}
  ∫_0^t\dd{s}α(t-s)\fdv{\ket{ψ({η}^\ast_t)}}{η^\ast_s}.
\end{equation}

There exist analytical approaches to this
term~\cite{Diosi1998Mar,Strunz2001Habil}, but we keep the approach as
general as possible and instead choose a numerical avenue.

They
key~\cite{Suess2014Oct,Hartmann2017Dec,Hartmann2021Aug,RichardDiss} is
to cast the complicated term containing the functional derivative into
an auxiliary state. Expanding the BCF into exponentials\footnote{See
  \cite{RichardDiss,Hartmann2021Aug} for a detailed account of the
  systematics of this expansion.}
\(α(τ)=∑_{μ}G_{μ=1}^{M}\eu^{-W_{μ}τ}\) and defining
\begin{equation}
  \label{eq:d_op_one}
  D_{μ}(t)\equiv ∫_{0}^{t} G_{μ} \eu^{-W_{μ}(t-s)} \fdv{η^\ast_s}\dd{s},\; D^{\vb{k}}(t)\equiv \prod_{μ=1}^{M}\sqrt{\frac{k_{μ}!}{G_{μ}^{k_{μ}}}}
  \frac{1}{i^{k_{μ}}}\pqty{D_{μ}}^{k_{μ}}
\end{equation}
we can define the \(\vb{k}th\) hierarchy state
\begin{equation}
  \label{eq:d_op_hier}
   \ket{ψ^{\vb{k}}}\equiv D^{\vb{k}}\ket{ψ}.
\end{equation}

For this state the following equation of motion can be
derived~\cite{Suess2014Oct,Hartmann2021Aug}
\begin{equation}
  \label{eq:singlehops}\tag{HOPS}
  \ket{\dot{ψ}^{\vb{k}}} = \qty[-\iu H_\sys + \vb{L}\cdot\vb{η}^\ast -
  ∑_{μ=1}^{M}k_{μ}W_μ]\ket{ψ^{\vb{k}}} +
  \iu ∑_{μ=1}^{M}\sqrt{G_μ}\qty[\sqrt{k_{μ}}  L \ket{ψ^{\vb{k} -
    \vb{e}_{μ}}} + \sqrt{\qty(k_{μ} + 1)}  L^† \ket{ψ^{\vb{k} +
    \vb{e}_{μ}}} ],
\end{equation}
where \(\vb{k}=(k_{1}, k_{2}, \ldots, k_{M})\) with \(k_{μ}\geq 0\) is
a multi index and \(\pqty{\vb{e}_{μ}}_{ν} = δ_{μ,ν}\). The term
\({\vb{k} - \vb{e}_{μ}}\) is evaluated only if \(k_{μ}\geq 1\). The
norm of the multi index \(\vb{k}\) is denoted
\(\abs{\vb{k}}=∑_{μ}k_{μ}\) and called the hierarchy level of
\(\ket{ψ^{\vb{k}}}\). The state
\(\ket{ψ(η_{t}^\ast, t)}\equiv \ket{ψ^{\vb{0}}}\) corresponds to the
trajectory obeying \cref{eq:nmqsd_single}.


We call \cref{eq:singlehops} the \emph{Hierarchy of Pure States}
because each hierarchy state couples only to the hierarchy states one
level above and one level below. This is similar to the
\emph{Hierarchical Equations of Motion} (HEOM) approach used
in~\cite{Kato2016Dec}, but with the advantage of reducing the
dimensionality from \(\dim{\hilb_{\sys}}^{2}\) to
\(\dim{\hilb_{\sys}}\) by treating pure states instead of density
matrices.

By truncating the hierarchy we obtain from \cref{eq:singlehops} a
system of time linear differential equations with time dependent
coefficients that can be solved numerically. By choosing a suitable
cutoff the method can be made arbitrarily
exact~\cite{Hartmann2021Aug}. For concrete truncation schemes see~
\cite{RichardDiss,Zhang2018Apr,Hartmann2021Aug} and
\cref{sec:truncsch}. In most simulations in this work the so called
\emph{simplex truncation scheme} was used. This scheme amounts to only
including hierarchy states with \(\abs{\vb{k}} \leq k_{\max}\).

Note that the hierarchy states have no physical interpretation, but
can be thought of as the ``memory'' of the system. We will find in
\cref{chap:flow} that they can be of use beyond the mere calculation
of \(\ket{ψ}=\ket{ψ^{\vb{0}}}\).

The nonlinear NMQSD \cref{eq:nmqsd_nonlin_single} can be accommodated
in much the same way, except for the replacements
\(L^\dag\rightarrow \pqty{L^\dag-\ev{L^\dag}_{t}}\) and
\(η\rightarrow \tilde{η}\) in \cref{eq:singlehops} transforming it
into a \emph{nonlinear} system of differential
equations~\cite{Suess2014Oct}. Throughout this work, the nonlinear
method is being used, as it offers much superior convergence.

\section{A Note on Tooling}
\label{sec:note-tooling}

The source codes for the simulations presented in this work span
multiple repositories. They all make heavy use of
\texttt{Numpy}~\cite{harris2020array} and
\texttt{Scipy}~\cite{2020SciPy-NMeth}. Further the
\texttt{Arb}~\cite{Johansson2017arb} is employed for its excellent
implementation of special functions. A small contribution was made to
\texttt{Arb}s \texttt{python} bindings in the course of this work.

The main code repository for this work can be found under
\url{https://github.com/vale981/master-thesis}.

The directory \path{python/energy_flow_proper} contains several
project subdirectory with literate programming notebooks in the
\texttt{org} format~\cite{EricSchulte2022Sep}. A detailed listing
linking subprojects to chapters can be found in
\cref{tab:code_structure}.

\begin{table}[htp]
  \centering
  \begin{tabular}{ll}
    Directory & Application \\
    \midrule
    \path{07_one_bath_systematics} & \cref{sec:prec_sim} \\
    \path{08_dynamic_one_bath} & \cref{sec:singlemod} \\
    \path{09_dynamic_two_bath_one_qubit} & \cref{sec:otto} \\
    \path{10_antizeno_engine} & Reproduction of~\cite{Mukherjee2020Jan}, not in this work. \\
    \path{11_new_ho_comparison} & \cref{sec:hopsvsanalyt}
  \end{tabular}
  \caption{\label{tab:code_structure}\small The subprojects in the main
    repository and their application.}
\end{table}

Code implementing the results of \cref{chap:flow,chap:analytsol} can
be found under \url{https://github.com/vale981/hopsflow}. The models
discussed in \cref{chap:numres,sec:therm_results} are implement in
\url{https://github.com/vale981/two_qubit_model}.

Some modifications were made to the code that generates the stochastic
processes for HOPS. These modifications can be found under
\url{https://github.com/vale981/stocproc}, although they are mostly
merged into the original repository
\url{https://github.com/cimatosa/stocproc} of Richard Hartmann
\orcidlink{0000-0002-8967-6183}.

The extensive and well tested existing HOPS code of the Theoretical
Quantum Optics group\footnote{Available upon reasonable request.} was
created by Richard Hartmann. Some improvements have been made in the
course of this work this work.  Documentation, type hints and unit
tests have been introduced. The parallelization mechanism was
overhauled and now uses the \href{https://www.ray.io}{\texttt{Ray}}
library. The structure of the code was further modularized, allowing
for a simple implementation of new integration methods. Time dependent
couplings and multiple baths were added to the implementation.

The source code of this document is available under
\url{https://github.com/vale981/master-thesis-tex}.
