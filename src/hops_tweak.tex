\chapter{Some Notes on HOPS}
\label{chap:hops_notes}
\section{Normalized HOPS}%
\label{sec:norm}

We introduce full HOPS vector \(Ψ = \qty(ψ, φ)\) which can be
decomposed into the zeroth hierarchy order state \(ψ\) and the
non-zero order states \(φ\).

The HOPS equations can then be written in an abstract manner as
\begin{equation}
  \label{eq:HOPS}
  \begin{aligned}
    \dot{ψ} &= F(ψ, φ), & \dot{φ} &= G(ψ, φ),
  \end{aligned}
\end{equation}
where \(c\cdot F(ψ, φ) = F(c\cdot ψ, c\cdot φ)\) and
\(c\cdot G(ψ, φ) = G(c\cdot ψ, c\cdot φ)\) for \(c\in\CC\)


The goal is to transform \(ψ \rightarrow \tilde{ψ}\) so that
\begin{equation}
  \label{eq:goal}
  \norm{\tilde{ψ}} = 1
\end{equation}
in a numerically stable manner.

Introducing the definitions \(\tilde{ψ} = \eu^{f(t)}ψ\) and
\(\tilde{φ} = \eu^{f(t)}φ\) with an
arbitrary\(f\colon \RR \rightarrow \CC\) we can begin to calculate
\begin{equation}
  \label{eq:normdgl}
  ∂_t\norm{\tilde{ψ}}^2 = \tilde{ψ}^† \qty(\dot{f}\tilde{ψ} +
  F(\tilde{ψ}, \tilde{φ})) + \cc = \dot{f} \abs{\tilde{ψ}}^2 +
  \tilde{ψ}^†F(\tilde{ψ}, \tilde{φ}) + \cc.
\end{equation}

We would now like to obtain \(∂_t\norm{\tilde{ψ}}^2 = 0\) as well as
\(\dot{f} > 0\) for \(\norm{\tilde{ψ}} < 1\), \(\dot{f} < 0\) for
\(\norm{\tilde{ψ}} > 1\) and \(\dot{f} = 0\) for
\(\norm{\tilde{ψ}}=1\), so that \(\norm{\tilde{ψ}} = 1\) becomes a
stable fix-point.

Observing \cref{eq:normdgl}, we conclude that our goals can be
achieved by demanding
\begin{equation}
  \label{eq:fdgl}
  \dot{f} = \frac{\tilde{ψ}^†F(\tilde{ψ},
    \tilde{φ})}{\norm{\tilde{ψ}}^2} + g\qty(\norm{ψ}^2)
\end{equation}
with \(g(0)=0\).

The first summand on its own would lead to norm conservation, \(∂_t\norm{\tilde{ψ}}^2 =
0\). The latter of our goals may be achieved by
choosing \(g(x) = \qty(1-x)\).

These choices lead to an altered HOPS equation
\begin{equation}
  \label{eq:normedhops}
  \dot{\tilde{Ψ}} = \qty[\frac{\tilde{ψ}^†F(\tilde{ψ},
    \tilde{φ})}{\norm{\tilde{ψ}}^2}+\qty(1-\norm{\tilde{ψ}}^2)]\mqty(\tilde{ψ}\\
  \tilde{φ}) + \mqty(F(\tilde{ψ},\tilde{φ}) \\ G(\tilde{ψ},\tilde{φ})).
\end{equation}

\section{Multiple Baths}
\label{sec:hops_multibath}

We generalize the NMQSD and HOPS to \(N\) baths for Hamiltonians of
the form~\cref{eq:multimodel}.


\subsection{NMQSD}
\label{sec:nmqsd}

Following the usual derivation of the NMQSD \cite{Diosi1998Mar}, we
switch to an interaction picture with respect to the \(H_\bath\)
leading to
\begin{equation}
  \label{eq:multimodelint}
  H(t) = H_\sys + ∑_{n=1}^N \qty[L_n^†B_n(t) + \hc],
\end{equation}
with \(B_n=∑_{λ} g_λ\nth a_λ\nth\eu^{-\iu ω_λ\nth t}\).

We will discuss the zero temperature case. The finite temperature
methods generalize straight forwardly to multiple baths.  Projecting
on a Bargmann (unnormalized) coherent state basis
\(\qty{\ket{\vb{z}^{(1)},\vb{z}^{(2)},\ldots}=
  \ket{\underline{\vb{z}}}}\) of the baths
\begin{equation}
  \label{eq:projected}
  \ket{ψ(t)} = ∫∏_{n=1}^N{\qty(\frac{\dd{\vb{z}\nth}}{π^{N_n}}\eu^{-\abs{\vb{z}}^2})}\ket{ψ(t,\underline{\vb{z}}^\ast)}\ket{\underline{\vb{z}}},
\end{equation}
where \(N_n\) are the number of oscillators in each bath.


We define
\begin{equation}
  \label{eq:processes}
  η^\ast_n(t) = {\qty(\vb{η}^\ast_t)}_n= -\iu ∑_λg_λ^{(n),\ast} z_λ^{(n),\ast}\eu^{\iu ω_λ\nth t}
\end{equation}
and using
\(\pdv{z_λ^{(n),\ast}}=∫\dd{s}\pdv{η^\ast_n(s)}{z_λ^{(n),\ast}}\fdv{η^\ast_n(s)}\)
we arrive at
\begin{equation}
  \label{eq:multinmqsd}
  ∂_tψ_t(\vb{η}^\ast_t) = -\iu H ψ_t(\vb{η}^\ast_t) +
  \vb{L}\cdot\vb{η}^\ast_tψ_t(\vb{η}^\ast_t) - ∑_{n=1}^N L_n^†∫_0^t\dd{s}α_n(t-s)\fdv{ψ_t(\vb{η}^\ast_t)}{η^\ast_n(s)},
\end{equation}
where \(α_n(t-s)= {\qty(\vb{α}(t-s))}_n=∑_λ\abs{g_λ\nth}^2\eu^{-\iu ω_λ\nth(t-s)}\) are the
zero temperature bath correlation functions. The equation
\cref{eq:multinmqsd} becomes the NMQSD by reinterpreting the
\(\vb{z}\nth\) as normal distributed complex random variables by
virtue of monte-carlo integration of \cref{eq:projected}. The
\(η^\ast_n(t)\) become homogeneous gaussian stochastic processes
defined through
\begin{equation}
  \label{eq:processescorr}
  \begin{aligned}
      \mathcal{M}(η^\ast_n(t)) &=0, & \mathcal{M}(η_n(t)η_m(s)) &= 0,
      & \mathcal{M}(η_n(t)η_m(s)^\ast) &= δ_{nm}α_n(t-s).
  \end{aligned}
\end{equation}

\subsection{Nonlinear NMQSD}
\label{sec:nonlin}

For the derivation of the lonlinear theory, the characteristic
trajectories of the partial differential equation of motion of
the Husimi-function
\begin{equation}
  \label{eq:husimi}
  Q_t(\underline{\vb{z}}, \underline{\vb{z}}^\ast) =
  \frac{\eu^{-\abs{{\underline{\vb{z}}}}^2}}{π^{∑_n N_n}}
  \braket{ψ(t, {\underline{\vb{z}}})}{ψ(t, {\underline{\vb{z}}}^\ast)}
\end{equation}
have to be determined.

Using \(∂_{\underline{\vb{z}}}\ket{ψ(t, {\underline{\vb{z}}}^\ast)} =
0\) and \(∂_{\underline{\vb{z}}^\ast}\bra{ψ(t, {\underline{\vb{z}}})} =
0\) because \(\ket{ψ(t, {\underline{\vb{z}}}^\ast)}\) is holomorphic
we derive
\begin{equation}
  \label{eq:husimimotion}
  ∂_tQ_t(\underline{\vb{z}}, \underline{\vb{z}}^\ast) = -i
  ∑_{n=1}^N\qty[∂_{z_λ^{(n), \ast}}\eu^{-\iu ω_λ\nth
    t}\ev{L^†_n}_tQ_t(\underline{\vb{z}}, \underline{\vb{z}}^\ast) - \cc],
\end{equation}
where \(\ev{L^†_n}_t = \mel{ψ(t, {\underline{\vb{z}}})}{L^†_n}{ψ(t,
  {\underline{\vb{z}}}^\ast)} / \braket{ψ(t, {\underline{\vb{z}}})}{ψ(t, {\underline{\vb{z}}}^\ast)}\).

The characteristics of \cref{eq:husimimotion} obey the equations of
motion
\begin{equation}
  \label{eq:characteristics}
  \dot{z}^{(n),\ast}_λ = \iu g_λ\nth \eu^{-\iu ω_λ\nth t} \ev{L^†_n}_t
\end{equation}
for the stochastic state labels.

The microscopic dynamics can in-turn be gathered into a shift of the
stochastic processes
\begin{equation}
  \label{eq:procshift}
  \tilde{η}_n^\ast(t) = η_n^\ast(t) + ∫_0^t\dd{s}α_n^\ast(t-s)\ev{L^†_n}_s
\end{equation}
and we obtain the nonlinear NMQSD equation
\begin{multline}
  \label{eq:multinmqsdnonlin}
  ∂_tψ_t(\tilde{\vb{η}}^\ast_t) = -\iu H ψ_t(\tilde{\vb{η}}^\ast_t) +
  \vb{L}\cdot\tilde{\vb{η}}^\ast_tψ_t(\tilde{\vb{η}}^\ast_t) \\-
  ∑_{n=1}^N
  \qty(L_n^†-\ev{L^†_n}_t)∫_0^t\dd{s}α_n(t-s)\eval{\fdv{ψ_t(\tilde{\vb{η}}^\ast_t)}{η^\ast_n(s)}}_{\vb{η}^\ast(s)
  = \vb{η}(\underline{\vb{z}}^\ast(t), s)}.
\end{multline}

The notation
\({\vb{η}^\ast(s) = \vb{η}(\underline{\vb{z}}^\ast(t), s)}\) means
that we replace the microscopic \(z_λ^{(n),\ast}\) in
\cref{eq:processes} with the shifted ones obeying
\cref{eq:characteristics} and evaluate the resulting function at \(s\).
This awkward construction can be remedied by the convolutionless
formulation. It plays no great role in the HOPS formalism.

\subsection{Multi Bath HOPS in Fock-Space Formulation}
\label{sec:multihops}

Following the usual derivation~\cite{RichardDiss} (but with a
different unitless normalization) and using an exponential expansion of the
BCFs \(α_n(τ)=∑_{\mu}^{M_n}=G_μ\nth\eu^{-W_μ\nth τ}\), we define
\begin{equation}
  \label{eq:dops}
  D_μ\nth(t) \equiv ∫_0^t\dd{s}G_μ\nth\eu^{-W_μ\nth (t-s)}\fdv{η^\ast_n(s)}
\end{equation}
and
\begin{equation}
  \label{eq:dops_full}
  D^{\underline{\vb{k}}} \equiv
  ∏_{n=1}^N∏_{μ=1}^{M_n}
  {\sqrt{\frac{\underline{\vb{k}}_{n,μ}!}{\qty(G\nth_μ)^{\underline{\vb{k}}_{n,μ}}}}
  \frac{1}{\iu^{\underline{\vb{k}}_{n,μ}}}}\qty(D_μ\nth)^{\underline{\vb{k}}_{n,μ}},
\end{equation}
as well as
\begin{equation}
  \label{eq:hierdef}
  ψ^{\underline{\vb{k}}} \equiv D^{\underline{\vb{k}}}ψ.
\end{equation}

Using
\begin{equation}
  \label{eq:commrelation}
  [D^\kmat(t),η_n^\ast(t)] =  \iu∑_{μ=1}^{M_n}
  \sqrt{\kmat_{n,μ}G\nth_μ} D^{\kmat -
    \mat{e}_{n,μ}}
\end{equation}
where \({\qty(\mat{e}_{n,μ})}_{ij}=δ_{ni}δ_{μj}\) we find after some algebra
\begin{multline}
  \label{eq:multihops}
  \dot{ψ}^\kmat = \qty[-\iu H_\sys + \vb{L}\cdot\vb{η}^\ast -
  ∑_{n=1}^N∑_{μ=1}^{M_n}\kmat_{n,μ}W\nth_μ]ψ^\kmat \\+
  \iu ∑_{n=1}^N∑_{μ=1}^{M_n}\sqrt{G\nth_μ}\qty[\sqrt{\kmat_{n,μ}}  L_nψ^{\kmat -
    \mat{e}_{n,μ}} + \sqrt{\qty(\kmat_{n,μ} + 1)}  L^†_nψ^{\kmat +
    \mat{e}_{n,μ}} ].
\end{multline}

The HOPS equations \cref{eq:multihops} can also be rewritten in an
especially appealing form \cite{Gao2021Sep} if we embed the hierarchy
states into a larger Hilbert space using
\begin{equation}
  \label{eq:fockpsi}
  \ket{Ψ} = \sum_\kmat\ket{\psi^\kmat}\otimes \ket{\kmat}
\end{equation}
where
\(\ket{\kmat}=\bigotimes_{n=1}^N\bigotimes_{μ=1}^{N_n}\ket{\kmat_{n,μ}}\)
are bosonic Fock-states.

Now \cref{eq:multihops} becomes
\begin{equation}
  \label{eq:fockhops}
  \begin{aligned}
    ∂_t\ket{Ψ} &= \qty[
                 \begin{aligned}
                 -\iu H_\sys + \vb{L}\cdot\vb{η}^\ast &-
                               ∑_{n=1}^N∑_{μ=1}^{M_n}b_{n,μ}^\dag b_{n,μ} W\nth_μ \\
                   &\qquad+
                 \iu ∑_{n=1}^N∑_{μ=1}^{M_n} \sqrt{G_{n,μ}} \qty(b^†_{n,μ}L_n +
                 b_{n,μ}L^†_n)
                 \end{aligned}
                 ] \ket{Ψ}\\
               &= \tilde{H}\ket{Ψ}
  \end{aligned}
\end{equation}

\section{Estimating the Norms of the Auxiliary States}
\label{sec:normest}

It is possible to find an (semi-rigorous) upper bound to the norms of
the auxiliary states. We will limit ourselves to one bath. The
generalization to multiple baths is straight forward.

Using \cref{eq:fockhops}, we can calculate
\begin{equation}
  \label{eq:normdiff}
  \begin{aligned}
    \iu ∂_t \norm{ψ^{\vb{k}}}^2
    &= \bra{Ψ}\ket{k}\bra{k}\tilde{H}\ket{Ψ} - \cc\\
    &= \qty(ψ^{\vb{k}})^†\bra{k}
      \qty[-\iu L η^\ast -\iu ∑_{μ=1}^{M}b_{μ}^\dag b_{μ} W_μ
      +∑_{μ=1}^{M} \sqrt{G_{μ}} \qty(b^†_{μ}L +
      b_μ L^†)]\ket{Ψ}- \cc\\
    &= \Bigg[-\iu \qty(ψ^{\vb{k}})^†L η^\ast ψ^{\vb{k}}
        -\iu ∑_{μ=1}^{M}k_μ W_μ \norm{ψ^{\vb{k}}}^2\\
        &\phantom{=}\quad -∑_{μ=1}^{M}\qty[\qty(ψ^{\vb{k}})^†\sqrt{G_{μ}k_μ}Lψ^{\vb{k}-\vb{e}_μ} +
        \qty(ψ^{\vb{k}})^†\sqrt{G_{μ}(k_μ+1)}Lψ^{\vb{k}+\vb{e}_μ} ]\Bigg]  - \cc.
  \end{aligned}
\end{equation}

We can now further treat the this expression to find the steady state
norms of the states.

Assuming generically that the term containing the stochastic process
\(η\) vanishes in the time average (as is the case for the steady
state) we will drop it in the following.

Terms of the form \(\Im(ψ^† O φ)\) may be estimated as follows
\begin{equation}
  \label{eq:genericest}
  \abs{\Im(ψ^† O φ)} \leq \norm{ψ} \norm{O φ} \leq \norm{ψ}\norm{O}\norm{φ},
\end{equation}
where the norm on the operator is the standard linear operator norm
\(\norm{O} = \max_{x\in \mathcal{H}}\frac{\ev{O}{x}}{\braket{x}}\).

We now endeavor to find from \cref{eq:normdiff} an estimate of the
steady state norm of \(ψ^{\vb{k}}\). To this end we assume that the
coupling to higher hierarchy states generically lowers the norm and is
therefore neglected. Using \cref{eq:genericest} we can estimate the
influence of the coupling to lower states, choosing the sign
so that the contribution to the norm is positive.

With this we obtain
\begin{equation}
  \label{eq:finalest}
  ∂_t \norm{ψ^{\vb{k}}}^2 = 0 = -∑_{μ=1}^{M}k_μ \Re[W_μ]
  \norm{ψ^{\vb{k}}}^2 +
  ∑_{μ=1}^{M}\abs{\sqrt{G_{μ}k_μ}}\norm{ψ^{\vb{k}}}\norm{ψ^{\vb{k}-\vb{e}_μ}}\norm{L}
\end{equation}
and therefore
\begin{equation}
  \label{eq:steadynorm}
  \norm{ψ^{\vb{k}}} =
  \frac{∑_{μ=1}^{M}\abs{\sqrt{G_{μ}k_μ}}\norm{ψ^{\vb{k}-\vb{e}_μ}}\norm{L}}{∑_{μ=1}^{M}k_μ \Re[W_μ]}.
\end{equation}


For the nonlinear method, the stochastic process obtains a shift whose
magnitude can be estimated as follows
\begin{equation}
  \label{eq:shiftestimate}
  \abs{η_{\mathrm{sh}}} \leq \norm{L} ∫_0^∞\dd{s}\abs{α^\ast(t-s)} \leq
  \norm{L} \sum_{μ=1}^M \frac{\abs{G_μ}}{\Re[W_μ]}.
\end{equation}
It is unclear how this shift should be treated. Simply adding it to
the denominator of~\cref{eq:steadynorm} lead to a breakdown of the
bound for numerical testing.  A better estimate should account for
this and also for the coupling to the lower orders foregoing the
recursive nature of the estimate.

The relation \cref{eq:steadynorm} is recursive
and break off at \(ψ^0\), the norm of which can be assumed to be \(1\)
in the nonlinear method.

These ideas remain to be verified. Especially the assumptions should
be checked. For time dependent coupling, one may maximize the estimate
over all \(L(t)\).

As an illustration the validity of the bound for one specific model
see \cref{fig:normest_ω,fig:normest_δ}.
\begin{figure}[t]
  \centering
  \plot{norm_est/norm_estimate_omega}
  \caption{\label{fig:normest_ω} The maximum norm (for each time
    point) of the first hierarchy states for \(500\) trajectories
    subtracted from norm estimate and normalized by the norm estimate
    for a qubit coupled to single zero temperature ohmic baths with
    varying \(ω_c\). The titles include the range of the norm
    estimates. For higher cutoff frequencies the BCF vanishes faster
    and the norm of the hierarchy states decreases. The opacity of the
    lines is proportional to their corresponding \(\abs{G_μ}\). The
    bound is tightest for the hierarchy states with the largest
    coupling. \fixme{reference model description}}
\end{figure}
\begin{figure}[p]
  \centering
  \plot{norm_est/norm_estimate_delta}
  \caption{\label{fig:normest_δ} The same as \cref{fig:normest_ω}, but
    for varying coupling strengths.}
\end{figure}

\subsection{Truncation Scheme}
\label{sec:truncsch}
The norm of the \(\vb{k}\)th hierarchy state scales like
\({1} / {\sqrt{\max_μk_μ}}\). This fact in itself, however, is not
too meaningful as the magnitude of the coupling to the lower hierarchy
states is
\begin{equation}
  \label{eq:couplingmag}
  M_{\vb{k}} = \norm{L} \norm{ψ^{\vb{k}}} \max_μ \abs{\sqrt{G_μk_μ}},
\end{equation}
which balances out the scaling.

Calculating \(M_{\vb{k}}\) explicitly and demanding it to be small
(compared to some energy scale) nevertheless gives a convergent
truncation scheme below a certain coupling strength.
Some basic experimentation has shown, that the cutoff parameter has to
be tuned and is not universally valid.
