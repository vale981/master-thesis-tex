\chapter{An Analytical Solution for Quantum Brownian Motion Models}
\label{chap:analytsol}

The results of \cref{chap:flow} are promising from a numerical
perspective remain to be verified. Previous
work~\cite{Hartmann2017Dec,RichardDiss} has made it clear that the
reduced system dynamics, but it is an open question whether bath
related quantities can be calculated to a similar degree of accuracy.

The best possible verification is the comparison with a soluble model,
ideally solved with a method completely different from the NMQSD. In
this chapter we will present the solution to the Heisenberg equations
of two quantum Brownian motion models for a single (\cref{sec:oneosc})
and two baths (\cref{sec:twoosc}) and time independent Hamiltonians.

These solutions will enable us to calculate the bath energy flow \(J\)
for one or two baths which will be compared with the numerical
solution in \cref{sec:hopsvsanalyt}.


\section{A Harmonic Oscillator coupled to a single Bath}
\label{sec:oneosc}
A simple quadratic model that is soluble~\cite{Breuer2002Jun} and of
the form \cref{eq:totalH} is given by
\begin{equation}
  \label{eq:one_ho_hamiltonian}
  H = \frac{Ω}{4}\qty(p^2+q^2) + \frac{1}{2} q
  \sum_λ\qty(g_λ^\ast b_λ + g_λ
  b^†_λ)+\sum_λ\omega_λ b^†_λ b_λ,
\end{equation}
where \(a,a^†\) are the ladder operators of the harmonic
oscillator, \(q=a+a^†\) and \(p=\frac{1}{\iu}\qty(a-a^†)\) so
that \([q,p] = 2\iu\).

The Heisenberg equations for \cref{eq:one_ho_hamiltonian}
\begin{align}
  \dot{q} &=Ω p \label{eq:qdot}\\
  \dot{p} &= -Ω q - \int_0^t \Im[α_0(t-s)] q(s)\dd{s} + W(t) \label{eq:pdot}
  \\
  \dot{b}_λ &= -\iu g_λ \frac{q}{2} - \iu\omega_λ b_λ
\end{align}
with the operator noise
\(W(t)=-\sum_λ \qty(g_λ^\ast b_λ(0)
\eu^{-\iu\omega_λ t } + g_λ b_λ^†(0)
\eu^{\iu\omega_λ t })\),
\(\ev{W(t)W(s)}=α(t-s)\) and \(α_0 \equiv \eval{α}_{T=0}\).

The equation \cref{eq:pdot} arises from
\begin{equation}
  \label{eq:bsol}
  b_λ(t) = b_λ(0) \eu^{-\iu ω_λ t} - \frac{\iu g_λ}{2}∫_0^t
  q(s) \eu^{-\iu ω_λ (t-s)}\dd{s}.
\end{equation}

The equations for \cref{eq:qdot} and \cref{eq:pdot} can be solved by
finding a matrix \(G(t)\) with \(G(0)=\id\) and
\begin{equation}
  \label{eq:eqmotprop}
  \dot{G}(t) = A G(t) - \int_0^t K(t-s) G(s)\dd{s},\quad A=\mqty(0 &
  Ω \\ -Ω & 0), \quad K(t)=\mqty(0 & 0\\ \Im[α_0(t)] & 0).
\end{equation}
Then
\begin{equation}
  \label{eq:qpsol}
  \mqty(q(t)\\ p(t)) = G(t)\mqty(q(0)\\ p(0)) + \int_0^tG(t-s)
  \mqty(0\\ W(s))\dd{s}.
\end{equation}

Because we are only interested in solutions for \(t\geq 0\) and the
shape of the convolution in \cref{eq:eqmotprop} the solution may be
found by virtue of the Laplace transform.

Setting
\begin{equation}
  \label{eq:laplprop}
  \mathcal{L}\{G\}(z) = \int_0^\infty \eu^{-z\cdot t} G(t)
\end{equation}
leads to an algebraic formula
\begin{equation}
  \label{eq:galgebr}
  \mathcal{L}\{G\}(z) = \qty(z-A + \mathcal{L}\{K\}(z))^{-1}.
\end{equation}

\subsection{Solution}
\label{sec:solution}
To solve \cref{eq:eqmotprop} and find the propagator \(G\), we have to
find an explicit expression for \cref{eq:galgebr}, a simple matrix
inversion, and then apply the inverse transformation.

We observe that
\begin{equation}
  \label{eq:mdef}
  M = z-A + \mathcal{L}\{K\}(z) = \mqty(z & -Ω\\ Ω +
  \mathcal{L}\{\Im[α_0]\}(z) & z)
\end{equation}
and therefore
\begin{equation}
  \label{eq:minv}
  M^{-1} = \frac{1}{Ω^2 + Ω\mathcal{L}\{\Im[α_0]\}(z) + z^2}
  \mqty(z & Ω \\ -(Ω + \mathcal{L}\{\Im[α_0]\}(z)) & z).
\end{equation}
From this we can conclude that \(G_{11}=G_{22}\).
Because \(\ev{W(s)}=0\) holds for thermal initial states of the bath we have
\begin{equation}
  \label{eq:meanvals}
  \mqty(\ev{q(t)}\\ \ev{p(t)}) = G(t)\mqty(\ev{q(0)}\\ \ev{p(0)}).
\end{equation}
Knowing this, we can deduce from \(\ev{\dot{q}}= Ω \ev{p}\) that
\begin{align}
  \label{eq:onlyoneneeded}
    G_{11} &= \frac{\dot{G}_{12}}{Ω} & G_{21} &=\frac{\ddot{G}_{12}}{Ω^2}.
\end{align}
These relations are true independent of the initial state of the
system. It therefore suffices if we concern ourselves with
\(G_{12}\).

We nevertheless continue in full generality and approach the inverse
Laplace transformation by expanding the BCF in terms of functions that
have a simple Laplace transform. As we also use an exponential
expansion in HOPS and are only interested in finite times, we may
choose \(α_0(t)=\sum_{n=1}^N G_n \eu^{-W_n t - \i \varphi_n}\) with
\(W_n=\gamma_n + \i\delta_n\) and
\(G_n, \varphi_n, \gamma_n,\delta_n\in\RR\) for \(t\geq 0\).

This leads to a mathematically simple expression for the Laplace
transform
\begin{equation}
  \label{eq:laplace_alpha}
  \mathcal{L}\qty{\Im[α_0]}(z) = -\sum_n G_n\qty[\frac{(z+\gamma_n)\sin\varphi_n+\delta_n\cos\varphi_n}{\delta_n^2+(\gamma_n+z)^2}].
\end{equation}
Because \(\mathcal{L}\{\Im[α_0]\}\) appears in the denominator of
\cref{eq:minv} it is desirable to write \cref{eq:laplace_alpha} with a
common denominator. Introducing \(s_n = \sin\varphi_n,\, c_n =
\cos\varphi_n\) and \(z_n= -W_k\) we arrive
at
\begin{equation}
  \label{eq:laplace_alpha_better}
  \begin{aligned}
  \mathcal{L}\qty{\Im[α_0]}(z) &= - \sum_n
  G_n\frac{(z_n+\gamma_n)s_n+ \delta_nc_n}{(z-z_n)(z-z_n^\ast)} \\
  &= -\frac{\sum_n G_n \qty((z_n+\gamma_n)s_n+
    \delta_nc_n)\prod_{k\neq
      n}(z-z_k)(z-z_k^\ast)}{\prod_{k}(z-z_k)(z-z_k^\ast)} \\
  &= \frac{\sum_n f_n(z)\prod_{k\neq n}(z-z_k)(z-z_k^\ast)}{\prod_{k}(z-z_k)(z-z_k^\ast)}
  \end{aligned}
\end{equation}
with the polynomials of first degree
\(f_n(z)=-G_n \qty((z_n+\gamma_n)s_n+\delta_nc_n)\).  Because the
above expression is a rational function, the components of
\cref{eq:minv} are rational functions for which the Laplace transform
is particularly simple to invert using the residue theorem. With this
in mind we now calculate
\begin{equation}
  \label{eq:prefactorrational}
      \frac{1}{Ω^2 + Ω\mathcal{L}\{\Im[α_0]\}(z) + z^2}
      % =\frac{\prod_{k}(z-z_k)(z-z_k^\ast)}{\qty[(z+\iΩ)(z-\iΩ)]\prod_{k}(z-z_k)(z-z_k^\ast)
      %   + \sum_nΩ f_n(z)\prod_{k\neq n}(z-z_k)(z-z_k^\ast)}\\
      =\frac{f_0(z)}{p_1(z) + \sum_n q_n(z)}
      =
      \frac{f_0(z)}{\mu\prod_{n=1}^{N+1}(z-\tilde{z}_l)(z-\tilde{z}^\ast_l)}
      = \frac{f_0(z)}{p(z)}
\end{equation}
where
\begin{align}
  f_0(z) &= \prod_{k}(z-z_k)(z-z_k^\ast) \\
  p_1(z) &= \qty[(z+\iΩ)(z-\iΩ)]\prod_{k}(z-z_k)(z-z_k^\ast) \\
  q_n(z) &= Ω f_n(z)\prod_{k\neq n}(z-z_k)(z-z_k^\ast)
\end{align}
and \(\mu\in\RR\). The \(\tilde{z}_l\) are the roots of the real
polynomial
\begin{equation}
  \label{eq:polyp}
  p(z) = p_1(z) + \sum_{n=1}^{N}q_n(z)
\end{equation}
of degree \(2(N+1)\) where we \textbf{assume that there are
  no roots with multiplicity greater than one}.
With this we can now calculate the inverse laplace transform of
expressions of the form \(\frac{f_0(z)g(z)}{p(z)}\) where \(g(z)\) is
any holonomic function so that \(\frac{f_0(z)g(z)}{p(z)} \eu^{z\cdot
  t}\) falls off fast enough for \(t\geq 0\),
\(\Re(z)>\max_l{\Re(\tilde{z}_l)}=\Delta\) and \(\Re(z) \rightarrow
-\infty\). Now the contour of the inverse Laplace
transform
\begin{equation}
  \label{eq:invlap}
  \mathcal{L}^{-1}\qty{\frac{f_0(z)g(z)}{p(z)}}(t) =
  \frac{1}{2\pi\i}\int_{\Delta - \i\infty}^{\Delta + \i\infty} \frac{f_0(z)g(z)}{p(z)} \eu^{z\cdot
  t}\dd{z}
\end{equation}
can be closed to the left to obtain
\begin{equation}
  \label{eq:simpleinvtrans}
  \mathcal{L}^{-1}\qty{\frac{f_0(z)g(z)}{p(z)}}(t)
  =
  \sum_{l=1}^{N+1}\qty[\frac{f_0(\tilde{z}_l)g(\tilde{z}_l)}{p'(\tilde{z}_l)}
  \eu^{\tilde{z}_l \cdot t} + \cc]
\end{equation}
where we assumed that \(g(z)^\ast=g(z^\ast)\) which is the case for
all our purposes. For completeness we give
\begin{equation}
  \label{eq:pderiv}
  p'(z) = 2\mu\sum_{k=1}^{N+1}\qty[(z-\Re(\tilde{z}_k))\prod_{\substack{n=1\\
    n\neq k}}^{N+1}(z-\tilde{z}_n)(z-\tilde{z}^\ast_n)].
\end{equation}

It can be immediately concluded that all elements of \(G\) are sums of
exponentials, just like the BCF. In particular
\begin{equation}
  \label{eq:gfinal}
  G(t) = \sum_{l=1}^{N+1}\qty[R_l \mqty(\tilde{z}_l & Ω \\ \frac{\tilde{z}_l^2}{Ω} & \tilde{z}_l)\eu^{\tilde{z}_l \cdot
    t} + \cc]
\end{equation}
with \(R_l={f_0(\tilde{z}_l)}/{p'(\tilde{z}_l)}\).

It may be noted that this solution does not contain any notion of
temperature, as we are working in the Heisenberg picture.

\subsubsection{Negative Times}
For completeness, it may be of interest to find a solution for
negative times. This solution is relatively unphysical, as the initial
condition of a product state plays a pivotal role in open system
dynamics~\cite{Rivas2012}. Therefore a system that starts out in some
entangled state just to reach the perfect product state at \(t=0\) is
not something that is likely to be applicable to physical questions.

The solution detailed above is only valid for positive times. Because
we strive to employ the same formalism again for negative times, we
will concern ourselves with the transformed quantities
\(\bar{X}(τ) = X(t(τ)) = X(-τ)\) so that \(τ ≥ 0\). It follows that
\(∂_τ \bar{X}(τ) = \bar{X}'(τ) = -\dot{X}(t(τ))\) so that
\begin{align}
  \bar{q}' &= -Ω \bar{p} \label{eq:qtag}\\
  \bar{p}' &= Ω \bar{q} + ∫_0^τ \Im[α_0(τ-s)] \bar{q}(s)\dd{s} - \bar{W}(τ) \label{eq:ptag}
  \\
  \bar{b}'_λ &= \iu g_λ \frac{q'}{2} + \iu\omega_λ b'_λ.
\end{align}
This leads to an equation for \(\bar{G}(τ)\), namely
\begin{equation}
  \label{eq:eqmotpropbar}
  \bar{G}'(τ) = -A \bar{G}(τ) + \int_0^τ K(τ-s) \bar{G}(s)\dd{s}.
\end{equation}
The solution is obtained from the \(t\geq 0\) case by substituting
\(A\rightarrow -A\) and \(K\rightarrow -K\).
We obtain for \(t\leq 0\)
\begin{equation}
  \label{eq:gfinalbar}
  \bar{G}(τ) = G(-τ) = G(t) = \sum_{l=1}^{N+1}\qty[R_l \mqty(\tilde{z}_l & -Ω \\ -\frac{\tilde{z}_l^2}{Ω} & \tilde{z}_l)\eu^{-\tilde{z}_l \cdot
    t} + \cc]
\end{equation}
and
\begin{equation}
  \label{eq:qpsolneg}
  \mqty(q(t)\\ p(t)) = G(t)\mqty(q(0)\\ p(0)) - \int_t^0 G(t-s)
  \mqty(0\\ W(s))\dd{s}.
\end{equation}


\subsection{Applications}
\label{sec:applications}
Having found an expression for \(G\), we have in principle solved the
model. It remains however to apply that solution in a way that is
contributing towards our goal of validating the results of
\cref{chap:flow}.


Knowing \(G\) and \(α\), we can calculate all observables of the
system with the ultimate goal of finding an expression for
\(J=-∂_{t}\ev{H_{B}}\). Simple closed form expressions of sums of
exponentials will be obtained by using an exponential expansion of
\(α\).

Throughout, we assume a thermal bath initial state so that
\(\ev{W(t)}=0\).

\subsubsection{Correlation Functions}
\label{sec:correl}

We proceed to calculate \(\ev{q(t)q(s)}\). For brevity we set
\(A=G_{11}\), \(B=G_{12}\), \(p_0=p(0)\) and \(q_0=q(0)\). Then,
\begin{equation}
  \label{eq:qcorrel}
  \ev{q(t)q(s)} =
  \begin{aligned}[t]
    & A(t)A(s) \ev{q_0^2} + B(t)B(s) \ev{p_0^2} +
    A(t)B(s)\ev{q_0p_0} + B(t)A(s)\ev{p_0q_0} \\
    & +\underbrace{∫_0^t\dd{l}∫_0^s\dd{r} B(t-l)B(s-r)α(l-r)}_{\equiv Λ(t, s)}.
  \end{aligned}
\end{equation}

For a Fock type initial state \(\ket{n}\) we have
\begin{equation}
  \label{eq:hoexp}
  \begin{aligned}
    \ev{q^2_0} &= \ev{p^2_0} = 2n+1 & \ev{q_0p_0} &= \iu.
  \end{aligned}
\end{equation}
Note that \(p\) and \(p\) differ from the usual definition by a factor
of \(2\).

\subsubsection{Bath Enery Derivative}
\label{sec:bathflow}
With \cref{eq:qcorrel} we can calculate the time derivative of the
bath energy expectation value
\begin{equation}
  \label{eq:bathderiv_1}
  \begin{aligned}
    \ev{\dot{H}_B} &= ∑_λ ω_λ \qty(\ev{b_λ^†\dot{b}_λ} + \cc) \\
    &=\frac{1}{4}∫_0^t\dd{s}\qty[\ev{q(s)q(t)} ∑_λ\abs{g_λ}^2 \eu^{\i
      ω_λ(t-s)} + \cc] +
    \i ∫_0^t\dd{s} G_{12}(s)∑_λ\abs{g_λ}\bar{n}_λ\qty[\eu^{\i ω_λ s}+\cc]\\
    &= -\frac{1}{2}\Im\qty[∫_0^t\dd{s}\ev{q(t)q(s)}\dot{α}_0(t-s)] +
    \frac{1}{2}∫_0^t\dd{s} G_{12}(s)\partial_s\qty[α(s)-α_0(s)] \\
    &=
    \begin{aligned}[t]
    -\frac{1}{2}\Im&\qty[∫_0^t\dd{s}\ev{q(t)q(s)}\dot{α}_0(t-s)] \\&+
    \frac12 G_{12}(t)[α(t)-α_0(t)]
    -\frac{Ω}{2}∫_0^t\dd{s} G_{11}(s)\qty[α(s)-α_0(s)],
    \end{aligned}
  \end{aligned}
\end{equation}
where we've used \(\ev{b_λ(0)}=\ev{b_λ^0}=0\),
\begin{equation}
  \label{eq:blambdadotexp}
  \begin{aligned}
    \ev{b_λ^†\dot{b}_λ}= -\i\ev{b_λ^†\qty(\frac{g_λ}{2}q + ω_λb_λ)} =
    -\i\qty[\frac{g_λ}{2}\ev{b^†_λ(t)q(t)} +
    \underbrace{ω_λ\ev{b^†_λ(t)b_λ(t)}}_{\in\RR\implies\text{cancels }\cc}].
  \end{aligned}
\end{equation}
and
\begin{equation}
  \label{eq:moreladot}
  \begin{aligned}
    \ev{b^†_λ(t)q(t)} &= \ev{\qty(b(0)^{\dag}_λ\eu^{\i ω_λ t} +
      \frac{\i}{2}∫_0^tg_λ^\ast q(s)\eu^{\i ω_λ (t-s)}\dd{s})q(t)} \\
    &= \frac{\i g_λ^\ast}{2}∫_0^t\ev{q(s)q(t)\eu^{\i ω_λ(t-s)}}\dd{s}
    - g_λ^\ast\bar{n}_λ∫_0^t G_{12}(s)\eu^{\i ω_λ s}\dd{s}
  \end{aligned}
\end{equation}
with \(\bar{n}_λ=\ev{b^†_λ(0)b_λ(0)}\).

For further evaluation of \cref{eq:bathderiv_1} we have to calculate
\begin{equation}
  \label{eq:lambdafold}
  \begin{aligned}
    Λ(t)&=∫_0^t\dd{s}Λ(t,s)\dot{α}_0(t-s)
    =∫_0^t\dd{s}∫_0^t\dd{l}∫_0^s\dd{r}
    B(t-l)B(s-r)α(l-r)\dot{α}_0(t-s)\\
    &=∫_0^t\dd{s}∫_0^t\dd{r}
    \begin{aligned}[t]
      Θ(s-r)&B(s-r)\dot{α}_0(t-s)\times\\
      \biggl[
      &∫_0^{t-r}\dd{u}B(t-r-u)α(u)+∫_0^{r}\dd{u}B(t-r+u)α^\ast(u)
      \biggr]
    \end{aligned}
    \\
    &= ∫_0^t\dd{r} g_1(t-r)\qty[g_2(t-r) + g_3(t,r)] = Λ_1(t) + Λ_2(t),
  \end{aligned}
\end{equation}
where \(Λ(t,s)\) was defined in \cref{eq:qcorrel}.

This expression now only uses \(α(t)\) for \(t\geq 0\) so that we can
once again employ the exponential expansion for \(α\).

We will arrive at expressions that are weighted sums of exponentials
whose detailed calculation is quite tedious and can be found
in\cref{sec:explicit_flow}.

\section{Two coupled Harmonic Oscillators coupled to two Baths}%
\label{sec:twoosc}
\begin{itemize}
\item (future) applications involve thermodynamically interesting case
  of multiple baths
\item we chose two HOs because then the coupling to the two baths is
  not trivially additive and we can reuse the method from
  \cref{sec:oneosc} without alteration
\end{itemize}
As we would like to verify our method also for more than one bath, a
model with two baths is required.

The considerations of~\cref{sec:oneosc} can be straight forwardly
generalized to the case of two coupled oscillators coupled in turn to
a bath each. This construction is chosen so that the previous results
can be reused and the coupling to the baths is not trivial.

We will not give explicit formulas for the results in terms of sums of
exponentials, as they are quite extensive and easily obtained via the
use of a computer algebra system or the aforementioned code.

The model is again given by a quadratic Hamiltonian
\begin{equation}
  \label{eq:hamiltonian_two_bath}
  \begin{aligned}
  H &= ∑_{i\in\qty{1,2}} \qty[H^{(i)}_O + q_iB^{(i)} + H_B^{(i)}] + \frac{γ}{4}(q_1-q_2)^2,
  \end{aligned}
\end{equation}
where \(H_O^{(i)}= \frac{Ω_i}{4}\qty(p_i^2+q_i^2)\), \(B^{(i)}=\sum_λ\qty(g^{(i),\ast}_λb^{(i)}_λ  + g^{(i)}_λ
  b^{(i),†}_λ)\) and \(H_B^{(i)}=\sum_λ\omega_λ b^{(i),†}_λ b^{(i)}_λ\).

The \(b^{(i)}\) are the usual bosonic ladder operators of the baths.
The \(a_i^{(i)},a_i^{†}\) are the ladder operators of the harmonic
oscillators and \(q_i=a_i+a_i^†\) and \(p=\frac{1}{\iu}\qty(a_i-a_i^†)\) so
that \([q_i,p_j] = 2\iuδ_{ij}\) and \([q_i,q_j] = [p_i,p_j] = 0\).

The Heisenberg equations for \cref{eq:hamiltonian_two_bath} are
\begin{align}
  \dot{q}_i &= Ω_i p_i \label{eq:qidot}\\
  \dot{p}_i &= -(Ω_i+γ) q_i - \int_0^t \Im[α^{(i)}_0(t-s)] q_i(s)\dd{s} + W_i(t) \label{eq:pidot}
  \\
  \dot{b}^{(i)}_λ &= -\iu g^{(i)}_λ \frac{q_i}{2} - \iu\omega^{(i)}_λ b^{(i)}_λ,
\end{align}
with the operator noise
\(W_i(t)=-\sum_λ \qty(g_λ^{(i),\ast} b^{(i)}_λ(0)
\eu^{-\iu\omega^{(i)}_λ t } + g_λ^{(i)} b_λ^{(i),†}(0)
\eu^{\iu\omega^{(i)}_λ t })\) satisfying \(\ev{W_i(s)}=0\) and
\(\ev{W_i(t)W_j(s)}=δ_{ij}α^{(i)}(t-s)\). We introduced
\(α^{(i)}_0 \equiv \eval{α^{(i)}}_{T=0}\).

We have given most quantities an extra index and accounted for the
coupling between the two oscillators. Apart from this, the equations
of motion have the same structure as in \cref{sec:oneosc}.

\subsection{Solution}
\label{sec:eqmot_two}
With the same general program as before, we we first obtain
\begin{equation}
  \label{eq:bsoltwo}
  b^{(i)}_λ(t) = b^{(i)}_λ(0) \eu^{-\iu ω^{(i)}_λ t} - \frac{\iu g^{(i)}_λ}{2}∫_0^t
  q_i(s) \eu^{-\iu ω^{(i)}_λ (t-s)}\dd{s}.
\end{equation}

We can solve the equations for the \(q_i,\,p_i\)
by finding a matrix \(G(t)\) with \(G(0)=\id\) and
\begin{gather}
  \label{eq:eqmotproptwo}
  \dot{G}(t) = A G(t) - \int_0^t K(t-s) G(s)\dd{s}\\
  A = \mqty(
  0 & \Omega  & 0 & 0 \\
  -\gamma -\Omega  & 0 & \gamma  & 0 \\
  0 & 0 & 0 & \Lambda  \\
  \gamma  & 0 & -\gamma -\Lambda  & 0),\;
  K(τ) =
  \mqty(0 & 0 & 0 & 0 \\
  \Im[α^{(1)}_0(t-s)] & 0 & 0 & 0 \\
  0 & 0 & 0 & 0 \\
  0 & 0 & \Im[α^{(2)}_0(t-s)] & 0),
\end{gather}
where \(Ω=Ω_1\) and \(Λ=Ω_2\) for convenience.

Then
\begin{equation}
  \label{eq:qpsol}
  \mqty(q_1(t)\\ p_1(t)\\ q_2(t)\\ p_2(t)) = G(t)\mqty(q_1(0)\\ p_1(0) \\ q_2(0)\\ p_2(0)) + ∫_0^tG(t-s)
  \mqty(0\\ W_1(s)\\ 0 \\ W_2(s))\dd{s}.
\end{equation}

With the Laplace transform find for \(t\geq 0\) the formula
\cref{eq:galgebr} for the Laplace transform of the solution, albeit
now with a more complicated matrix
\begin{equation}
  \label{eq:mdeftwo}
  \begin{aligned}
    M &= z-A + \mathcal{L}\{K\}(z)\\
    &= \mqty(z & -\Omega  & 0 & 0 \\
    \mathcal{L}\{\Im[α^{(1)}_0]\}(z)+\gamma +\Omega  & z & -\gamma  & 0 \\
    0 & 0 & z & -\Lambda  \\
    -\gamma  & 0 & \mathcal{L}\{\Im[α^{(2)}_0]\}(z)+\gamma +\Lambda  & z)
  \end{aligned}
\end{equation}
that we have to invert.

This can be done easily\footnote{We have use a computer algebra
  system. There is probably a pattern to the inverse matrix, so that
  the solution for \(N>2\) oscillators may be found.}  and yields
\begin{equation}
  \label{eq:minvtwo}
  M^{-1}(z) = \frac{1}{\det[M](z)} \tilde{M}(z)
\end{equation}
where \(\tilde{M}\) is a matrix containing only polynomials of \(z\)
and of the Laplace transforms of the bath correlation functions.

The numerator is
\begin{equation}
  \label{eq:numerator}
  \begin{aligned}
  \det[M](z)=a(z)& b(z) \Lambda  \Omega +a(z) \left(\gamma
    \Lambda  \Omega +\Lambda ^2 \Omega +\Omega  z^2\right)
  +b(z)
  \left(\gamma \Lambda  \Omega +\Lambda  \Omega ^2+\Lambda  z^2\right)\\
  &+\gamma  \Lambda ^2 \Omega +\gamma  \Lambda
   \Omega ^2+\Lambda ^2 \Omega ^2+\gamma  \Lambda  z^2+\gamma  \Omega  z^2+\Lambda ^2 z^2+\Omega ^2 z^2+z^4,
  \end{aligned}
\end{equation}
where \(a(z)=\mathcal{L}\{\Im[α^{(1)}_0]\}(z)\) and
\(b(z)=\mathcal{L}\{\Im[α^{(2)}_0]\}(z)\).

Using the same approach as in \cref{sec:solution}, we arrive at an
expression similar to \cref{eq:prefactorrational} for
\((\det[M](z))^{-1}\). The polynomial \(p\) is now of degree
\(4 + 2 \qty(N^{(1)} + N^{(j)})\) where the \(N^{(i)}\) are the number of
terms in the expansions of bath correlation functions for each bath
and the function \(f_0\) now depends on both bath correlation
functions.

We ultimately find that \(G\) is a sum of
exponentials
\begin{equation}
  \label{eq:gfinal}
  G(t) = \sum_{l=1}^{2+N_1+N_2}\qty[R_l \tilde{M}(\tilde{z}_l)\eu^{\tilde{z}_l \cdot
    t} + \cc]
\end{equation}
with \(R_l={f_0(\tilde{z}_l)}/{p'(\tilde{z}_l)}\) as defined in
\cref{sec:solution}.

\subsection{Applications}
\subsubsection{Correlation Functions}
\label{sec:correltwo}
We can now proceed to calculate the correlation functions
\(C(t,s) = \ev{x_i(t)x_j(s)}\) where the \(x_i\) are the phase space operators
of the two harmonic oscillators. This will enable use to calculate the
system energies of the two oscillators (omitted here) and again the
bath energy flows of the two baths.

We find
\begin{equation}
  \label{eq:generalcorr}
  C_{ij}(t, s) = G_{ik}(t)G_{jl}(s) C(0,0)_{kl} +
  \underbrace{∫_0^t\dd{l}∫_0^s\dd{r}G_{ik}(t-l)G_{jl}(s-r) \ev{W_k(l)W_l(r)}}_{=Θ_{ij}}.
\end{equation}

The matrix \(Θ_{ij}\) contains the bath-induced correlations and can
be calculated as in the single-oscillator case.

For two oscillators that are initially in Fock states
\(\ket{ψ_{0}}=\ket{n}\otimes\ket{m}\) we have
\begin{equation}
  \label{eq:initial_corr}
  C(0,0) =
  \begin{pmatrix}
    1 + 2 n & \i & 0 & 0 \\
    -\i & 1 + 2 n & 0 & 0 \\
    0 & 0 & 1 + 2 m & \i \\
    0 & 0 & -\iu & 1 + 2m
  \end{pmatrix}.
\end{equation}

\subsubsection{Bath Enery Derivative}
\label{sec:bathflowtwo}

Similar to the calculations in \cref{sec:bathflow} we find
\begin{equation}
  \label{eq:bathderivtwo}
  \ev{\dot{H}_B^{(n)}}=-\frac12
  \Im∫_0^tC_{2n-1, 2n-1}(t,s)\dot{α}_0^{(n)}(t-s)\dd{s} + \frac12 ∫_0^t
  ∑_{k=1,2}G_{2n-1,2k}∂_s\qty(α^{(k)}(s)-α_0^{(k)}(s)).
\end{equation}

This can be evaluated using the exponential expansions and yields yet
another sum of exponentials. The steady state flow can then be found
be setting all exponentials to zero, although care has to be taken, as
an exponential fit of the BCF may be only valid for finite times.

This concludes the calculation. Python code implementing the solution
can be found under \url{https://github.com/vale981/hopsflow}.
