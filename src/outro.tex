\chapter{Conclusion and Ideas for future Work}
\label{cha:concl-ideas-future}
A worthwhile task for future work would be to verify the results
summarized in \refcite{Binder2018} for the Otto cycle. Especially the
optimization for optimal power which leads to the
Novikov–Curzon–Ahlborn efficiency \(η_{ca}=1-\sqrt{T_{c}/T_{h}}\) is
interesting in the case of stronger coupling.

Another cycle to study would be a Carnot-type cycle, where the
modulation of the system and the thermalization with the bath occur at
the same time. Interpolating between Otto and Carnot, as well as
studying the effect of overlapping and shifting strokes is a
fascinating avenue for future exploration.

Also, more interesting working media, such as a three level system are
of interest. In \refcite{Uzdin2015Sep} it is shown, that in certain
regimes quantum coherence can lead to superior power output. In the
same regime different types heat engines are equivalent. Both these
effects have been observed experimentally in \refcite{Klatzow2019Mar}. It
would be interesting to see if the slight deviations from theory in
\cite{Klatzow2019Mar} could be explained using HOPS.

The so called Anti-Zeno Effect occurring in systems under fast
modulation has recently received some attention
\cite{Mukherjee2020Jan,Xu2022Mar}. An advantage is claimed to exist,
due to the broadening of the resonance criterion which we have
observed in
\cref{sec:one_bath_cutoff,sec:modcoup_reso,sec:otto}. Being a
consequence of the energy time uncertainty it is being argued, that
the origin of this advantage is truly quantum. The tools for the
exploitation of this effect and its verification are provided in this
work. However, a strong coupling analysis has already been performed
using HEOM in \refcite{Xu2022Mar}.

In \refcite{Santos2021Jun} a cycle is proposed that first creates states
of finite ergotropy by letting energy flow through the working medium
and then extracting this ergotropy in a separate stroke. This work
could be verified and expanded to the non-Markovian regime.

A useful improvement of the method would be the ability to snapshot
the total state of system and bath and then propagate this state with
different modulation protocols. Also, exploring the thermofield method
for finite temperature to avoid the slow convergence of the flow may
be worthwhile. However, at least for coupling that is not hermitian,
this would only trade computational effort for memory, as the number
of hierarchy states would increase.

Finally, in the spirit of~\cite{Esposito2015Dec} one could employ the
HOPS to verify whether a given definition of internal energy that
includes the interaction energy is path independent.
