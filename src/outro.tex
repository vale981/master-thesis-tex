\chapter{Conclusion and Outlook}
\label{cha:concl-ideas-future}

In this work, we set out to find a way of accessing bath related
observables, such as the expected bath energy change and the
interaction energy expectation value using the
NMQSD\footnote{Non-Markovian Quantum State
  Diffusion}/HOPS\footnote{Hierarchy of Pure States} framework which
we introduced in \cref{chap:intro}.

This endeavour was indeed successful, as we laid out in
\cref{chap:flow}. The crucial point is, that we still have access to
the bath degrees of freedom \emph{before} performing a stochastic
average over the NMQSD/HOPS trajectories. This allows us to compute
the expectation values of observables which contain the bosonic bath
operators collectively.

In \cref{chap:flow} we presented an analytic solution to a well known
model for quantum Brownian motion. Using this solution, we were able
to derive expressions for the bath energy change
\(∂_{t}\ev{H_{\bath}}\) for models with one and two baths.

This enabled us to verify the results of \cref{chap:flow} in
\cref{chap:numres} by solving the same model numerically using
HOPS. Excellent agreement was found in \cref{sec:hopsvsanalyt}.

Turning to the spin-boson model in \cref{sec:prec_sim}, we used energy
conservation to verify again, that we can consistently and efficiently
compute bath related observables with HOPS upon the example of the
interaction energy expectation value. In the cases where the
consistency condition was not met, we nevertheless found that the
results were qualitatively correct. When choosing less precise
numerical parameters, the direct calculation of the interaction energy
by the use of \cref{sec:intener} yields results that are generally
much more precise than the ones obtained through energy conservation.

We continued to explore the energy transfer behavior of the zero
temperature spin-boson model and found that energy transfer
performance for strong coupling has a complicated dependence on the
shape of the spectral density of the bath. Energy transfer performance
can be optimized through longer bath memories and resonant baths when
the interaction is switched off at the right time. Switching the
interaction off in finite time leads to cooling of the system,
especially when the steady state had been reached.

Having explained short time dynamics of the bath energy change in
\cref{sec:pure_deph} neglecting the system Hamiltonian, which we
verified for the spin-boson model in
\cref{sec:initial-slip-sb,sec:moder-init-slip}. It was further found,
that this short time behaviour is already present on the trajectory
level so that there are no stochastic fluctuations for short
times. Instead, the auxiliary states of the HOPS are being populated
during this initial period.

In \cref{sec:singlemod} we turned to some issues of quantum
thermodynamics. We reviewed general analytical results that bounded
energy extraction from open systems in \cref{sec:basic_thermo}, both
for the single-bath and the multi-bath case. We then turned to some
more challenging applications of the HOPS method. First, a driven
spin-boson model was considered. We found that a not insignificant
fraction of the upper bound on the ergotropy can be extracted by
modulating the coupling and providing a bath with long memory time. We
also demonstrated quantum friction, a quantum speed limit and a bath
resonance phenomenon.

Finally, we treated a model with multiple baths and non-harmonic
smooth modulation in \cref{sec:otto}. A cyclic modulation protocol was
implemented upon a two level system coupled to two baths in a
spin-boson like fashion. We achieved finite power with finite
efficiency and verified a Gibbs-like inequality
\cref{sec:operational_thermo}. When disabling the coupling modulation,
the power and efficiency were much reduced.

A worthwhile task for future work would be to verify the results
summarized in \refcite{Binder2018} for the Otto cycle. Especially the
optimization for optimal power which leads to the
Novikov–Curzon–Ahlborn efficiency \(η_{ca}=1-\sqrt{T_{c}/T_{h}}\) is
interesting in the case of stronger coupling.

Another cycle to study would be a Carnot-type process, where the
modulation of the system and the thermalization with the bath occur at
the same time. Interpolating between Otto and Carnot, as well as
studying the effect of overlapping and shifting strokes is a
fascinating avenue for future exploration.

Also, more interesting working media, beginning with a three level
system, are of interest. In \refcite{Uzdin2015Sep} it is shown that
in certain regimes quantum coherence can lead to superior power
output. In the same regime different types heat engines are
equivalent. Both these effects have been observed experimentally in
\refcite{Klatzow2019Mar}. It would be interesting to see if the slight
deviations from theory in \cite{Klatzow2019Mar} could be explained
using HOPS.

The so called Anti-Zeno Effect occurring in systems under fast
modulation has recently received some attention
\cite{Mukherjee2020Jan,Xu2022Mar}. An advantage is claimed to exist,
due to the broadening of the resonance criterion which we have
observed in
\cref{sec:one_bath_cutoff,sec:modcoup_reso,sec:otto}. Being a
consequence of the energy time uncertainty it is argued that the
origin of this advantage is truly quantum. The tools for the
exploitation of this effect and its verification are provided in this
work. However, a strong coupling analysis has already been performed
using HEOM in \refcite{Xu2022Mar}.

In \refcite{Santos2021Jun} a cycle is proposed that first creates states
of finite ergotropy by letting energy flow through the working medium
and then extracting this ergotropy in a separate stroke. This work
could be verified and expanded to the non-Markovian regime.

A useful technical improvement of the method would be the ability to
snapshot the total state of system and bath and then propagate this
state with different modulation protocols. Also, exploring the
thermofield method for finite temperature to avoid slow convergence
would be helpful. However, at least for coupling that is not
hermitian, this would likely only trade computational effort for
memory, as the number of hierarchy states would increase.

The splitting up of the stochastic process to calculate, for example, the
energy change of parts of the bath as discussed at the end of
\cref{sec:general_obs} is also very interesting.

Finally, in the spirit of~\cite{Esposito2015Dec} one could employ the
HOPS to verify whether a given definition of internal energy that
includes the interaction energy is path independent.
