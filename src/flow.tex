\chapter{Bath Observables and the Hierarchy of Pure States}
\label{chap:flow}
\section{Linear NMQSD, Zero Temperature}
\label{sec:flow_lin}
As in~\cite{Hartmann2017Dec} we choose
\begin{equation}
  \label{eq:totalH}
  H = H_\sys + \underbrace{LB^† + L^† B}_{H_\inter} + H_\bath
\end{equation}
with the system hamiltonian \(H_\sys\), the bath hamiltonian
\begin{equation}
  \label{eq:bathh}
  H_\bath = ∑_\lambda ω_\lambda a^† a,
\end{equation}
the bath coupling system operator \(L\) and the bath coupling bath
operator
\begin{equation}
  \label{eq:bop}
  B=∑_{\lambda} g_{\lambda} a_{\lambda}
\end{equation}
which define the interaction hamiltonian \(H_\inter\).

We define the heat flow out of the system as in~\cite{Kato2015Aug}
through
\begin{equation}
  \label{eq:heatflowdef}
  J = - \dv{\ev{H_\bath}}{t}.
\end{equation}
Working, for now, in the Schr\"odinger picture the Ehrenfest theorem
can be employed to find
\begin{equation}
  \label{eq:ehrenfest}
  \i∂_t\ev{H_\bath} = \ev{[H_\bath,H]} = \ev{[H_\bath,H_\inter]}.
\end{equation}
Thus, we need to calculate
\begin{eqnarray}
  \label{eq:calccomm}
  \begin{aligned}
    [H_\bath,H_\inter] &= [H_\bath, LB^† + L^† B] \\
    &= L[H_\bath, B^† ] + L^† [H_\bath, B] \\
    &= L[H_\bath, B^† ] - \hc.
  \end{aligned}
\end{eqnarray}
This checks out as the commutator has to be anti-hermitian due to
\cref{eq:ehrenfest}.
Using \([H_\bath, B^† ]=∑_\lambda ω_\lambda g^\ast_\lambda
a^†_\lambda\) it follows that
\begin{equation}
  \label{eq:expcomm}
  \begin{aligned}
    \ev{[H_\bath,H_\inter]} &= ∑_\lambda ω_\lambda g^\ast_\lambda
    \ev{La^†_\lambda} - \cc
    = ∑_\lambda ω_\lambda g^\ast_\lambda
    \ev{La^†_\lambda \eu^{\i ω t}}_\inter - \cc\\
    &= \frac{1}{\i}\ev{L∂_t{∑_\lambda
        g^\ast_\lambda a^†_\lambda \eu^{\i ω t}}}_\inter - \cc
    =\frac{1}{\i}\qty(\ev{L\dot{B}^†}_\inter  + \cc)
  \end{aligned}
\end{equation}
where we switched to the interaction picture with respect to \(H_\bath\)
in keeping with the standard NMQSD formalism.
In essence this is just the Heisenberg equation for \(H_\inter\). The
expression for \(J\) follows
\begin{equation}
  \label{eq:final_flow}
  J(t) = \ev{L^†∂_t B(t) + L∂_t B^†(t)}_\inter.
\end{equation}

From this point on, we will assume the interaction picture and drop
the \(I\) subscript. The two summands yield different expressions in
terms of the NMQSD.  For use with HOPS with the final goal of
utilizing the auxiliary states the expression \(\ev{L^†∂_t B(t)}\)
should be evaluated. When considering the complex conjugate of this
expression, we find a formula involving the derivative of the driving
stochastic process. This is undesirable as it does not exist for all
bath correlation functions\footnote{Only for BCFs that are smooth at
  \(τ=0\).} and expressions involving the process directly are alleged
to converge slower. The last fact may be explained by the fact, that
one needs quite a lot of sample paths of the process for the mean of
those sample paths to converge to zero. On the other hand, the first
hierarchy states do contain an integral of-sorts of the sample paths
and are not as sensitive to fluctuations.

We calculate
\begin{equation}
  \label{eq:interactev}
  \ev{L^†∂_t B(t)}=\ev{L^†∂_t B(t)}{\psi(t)} =
  ∫ \braket{\psi(t)}{z}\mel{z}{L^†∂_tB(t)}{\psi(t)}\frac{\dd[2]{z}}{\pi^N},
\end{equation}
where \(N\) is the total number of environment oscillators and
\(z=\qty(z_{\lambda_1}, z_{\lambda_2}, \ldots)\).  Using
\(\mel{z}{a_λ}{ψ}= ∂_{z^\ast_λ}\braket{z}{ψ}=
∂_{z^\ast_λ}\ket{ψ(z^\ast,t)}\) and
\(\mel{z}{a_λ^\dag}{ψ}= z_λ^\ast\braket{z}{ψ}\) we find,
\begin{equation}
  \label{eq:nmqsdficate}
  \begin{aligned}
    \mel{z}{∂_tB(t)}{\psi(t)} &= ∑_\lambda g_\lambda
  \qty(∂_t \eu^{-\iω_\lambda
    t})∂_{z^\ast_\lambda}\ket{\psi(z^\ast,t)} \\
  &= ∫_0^t ∑_\lambda g_\lambda
  \qty(∂_t \eu^{-\iω_\lambda
    t})\pdv{η_s^\ast}{z^\ast_\lambda}\fdv{\ket{\psi(z^\ast,t)}}{η^\ast_s}\dd{s}\\
  &= -\i∫_0^t\dot{\alpha}(t-s)\fdv{\ket{\psi(η^\ast,t)}}{η^\ast_s}\dd{s},
  \end{aligned}
\end{equation}
where
\(η^\ast_t\equiv -\i ∑_\lambda g^\ast_\lambda z^\ast_\lambda
\eu^{\iω_\lambda t}\) which led to the chain rule
\(∂_{z^\ast_λ}=∫\dd{s}\pdv{η_s^\ast}{z^\ast_λ}\fdv{}{η_s^\ast}\). With
this we obtain
\begin{equation}
  \label{eq:steptoproc}
  \ev{L^†∂_t B(t)} = -\i \mathcal{M}_{η^\ast}\bra{\psi(η,
    t)}L^†∫_0^t\dd{s} \dot{\alpha}(t-s)\fdv{η^\ast_s} \ket{\psi(η^\ast,t)}.
\end{equation}
Defining
\begin{equation}
  \label{eq:defdop}
D_t = ∫_0^t\dd{s} \alpha(t-s)\fdv{η^\ast_s}
\end{equation}
as in~\cite{Suess2014Oct} we find
\begin{equation}
  \label{eq:final_flow_nmqsd}
  J(t) = -\i \mathcal{M}_{η^\ast}\bra{\psi(η,
    t)}L^†\dot{D}_t\ket{\psi(η^\ast,t)} + \cc,
\end{equation}
where we've used that the integral in \(D_t\) can be expanded over the
whole real axis. If we assume \(\alpha = \exp(-w t)\) then
\begin{equation}
  \label{eq:hopsj}
  J(t) = \i \mathcal{M}_{η^\ast}\bra{\psi^{(0)}(η,
    t)}wL^†\ket{\psi^{(1)}(η^\ast,t)} + \cc.,
\end{equation}
where \(\ket{\psi^{(1)}(η^\ast,t)}\) is the first HOPS hierarchy
state. This can be generalized to any BCF that is a sum of exponentials.

Interestingly one finds that
\begin{equation}
  \label{eq:alternative}
  \ev{L∂_t B^†(t)} = \i∫\frac{\dd[2]{z}}{\pi^N}
  \dot{η}_t^\ast \mel{\psi(η,t)}{L}{\psi(η^\ast,t)}.
\end{equation}
However, this approach becomes more complicated in the nonlinear
method.
The previous expression has the advantage
that we utilize the first hierarchy states that are already being
calculated as a byproduct.

In the language of~\cite{Hartmann2021Aug} we can generalize to
\(\alpha(t) = ∑_i G_i \eu^{-W_i t}\) and thus
\begin{equation}
  \label{eq:hopsflowrich}
  J(t) = ∑_\mu\frac{G_\mu W_\mu}{\bar{g}_\mu} \i\mathcal{M}_{η^\ast}\bra{\psi^{(0)}(η,
    t)}L^†\ket{\psi^{\vb{e}_\mu}(η^\ast,t)} + \cc,
\end{equation}
where \(\psi^{\vb{e}_\mu}\) is the \(\mu\)-th state of the first
hierarchy and \(\bar{g}_\mu\) is an arbitrary scaling introduced in
the definition of the hierarchy in~\cite{Hartmann2021Aug} to help with
the scaling of the norm.

With the new ``fock-space'' normalization (see \cref{eq:dops_full})
however the expression becomes
\begin{equation}
  \label{eq:hopsflowfock}
  J(t) = - ∑_\mu\sqrt{G_\mu}W_\mu
  \mathcal{M}_{η^\ast}\bra{\psi^{(0)}(η,
    t)}L^†\ket{\psi^{\vb{e}_\mu}(η^\ast,t)} + \cc.
\end{equation}

\section{Nonlinear NMQSD, Zero Temperature}
\label{sec:nonlin_flow}
In the spirit of the usual derivation of the nonlinear NMQSD we write
\begin{equation}
  \label{eq:newb}
  \begin{aligned}
  \ev{L^†\dot{B}(t)} &= ∫ \frac{\dd[2]{z}}{\pi^N} \eu^{-\abs{z}^2}
  \braket{\psi}{z}\!\braket{z}{\psi}
  \frac{\braket{\psi(t)}{z}\!\mel{z}{L^†\dot{B}(t)}{\psi(t)}}{\braket{\psi}{z}\!\braket{z}{\psi}}
  \\
  &= ∫ \frac{\dd[2]{z}}{\pi^N} \eu^{-\abs{z}^2}
  \frac{\mel{z(t)}{L^†\dot{B}(t)}{\psi(t)}}{\braket{\psi}{z(t)}\!\braket{z(t)}{\psi}},
  \end{aligned}
\end{equation}
where \(z_{\lambda}^{*}(t)=z_{\lambda}^{*}+\i g_{\lambda} ∫_{0}^{t}
\dd{s} \eu^{-\i ω_{\lambda} s}\ev{L^†}_{s}\).
We find that next steps are the same as in \cref{sec:nonlin} by noting
\begin{equation}
  \label{eq:deriv_trick}
  \begin{aligned}
  \eval{∂_{z^\ast_\lambda}}_{z^\ast=z_\lambda^\ast(t)} &=
  ∫_0^t\dd{s}\eval{\pdv{η^\ast_s}{z^\ast_\lambda}}_{z^\ast=z^\ast_\lambda(t)}
                                                         \fdv{}{η^\ast_s(z^\ast=z^\ast(t))} \\
    &=
  ∫_0^t\dd{s}\eval{\pdv{η^\ast_s}{z^\ast_\lambda}}_{z^\ast=z^\ast(0)}
  \fdv{}{η^\ast_s(z^\ast=z^\ast(t))},
  \end{aligned}
\end{equation}
which does alter the definition of \(D_t\) but results in the same
HOPS equations.
The shifted process \(\tilde{η}^\ast=
η^\ast(z^\ast(t),t)=η^\ast(t) +
∫_0^t\dd{s}\alpha^\ast(t-s)\ev{L^†}_{\psi_s}\) appears directly
in the NMQSD equation but results only in a slight change in the
functional derivative. Note however that
\begin{equation}
  \label{eq:fdvclarification}
  \fdv{}{η^\ast_s(z^\ast=z^\ast(t))} \neq \fdv{}{\tilde{η}^\ast_s}
\end{equation}
which is not problematic as we have (implicit in~\cite{Diosi1998Mar})
\begin{equation}
  \label{eq:fdvhops}
  \fdv{}{η^\ast_s(z^\ast=z^\ast(t))} \ket{\psi(z^\ast)} = \fdv{}{η^\ast_s}\ket{\psi(z^\ast(t, z^\ast_0), t)}
\end{equation}
so that the usual HOPS hierarchy follows. Note \(z^\ast_0 = z^\ast(0)\).

Therefore,
\begin{equation}
  \label{eq:newbcontin}
  J(t) =
  -\i
  \mathcal{M}_{\tilde{η}^\ast}\frac{\mel{\psi(\tilde{η},t)}{L^†\dot{\tilde{D}}_t}{\psi(\tilde{η}^\ast,t)}}{\braket{\psi(\tilde{η},t)}{\psi(\tilde{η}^\ast,t)}}
  + \cc,
\end{equation}
where the dependence on \(\tilde{η}\) is symbolic and to be
understood in the context of \cref{eq:fdvhops}.

Again we express the result in the language of~\cite{Hartmann2021Aug}
to obtain
\begin{equation}
  \label{eq:nonlinhopsflowrich}
  J(t) = ∑_\mu\frac{G_\mu W_\mu}{\bar{g}_\mu}
  \i\mathcal{M}_{η^\ast}\frac{\bra{\psi^{(0)}(η,
      t)}L^†\ket{\psi^{\vb{e}_\mu}(η^\ast,t)}}{\bra{\psi^{(0)}(η,
      t)}\ket{\psi^{0}(η^\ast,t)}} + \cc.
\end{equation}

With the new ``fock-space'' normalization however the expression
becomes
\begin{equation}
  \label{eq:nonlinhopsflowfock}
  J(t) = - ∑_\mu\sqrt{G_\mu}W_\mu
  \mathcal{M}_{η^\ast}\frac{\bra{\psi^{(0)}(η,
      t)}L^†\ket{\psi^{\vb{e}_\mu}(η^\ast,t)}}{\bra{\psi^{(0)}(η,
      t)}\ket{\psi^{0}(η^\ast,t)}} + \cc.
\end{equation}

\section{Linear Theory, Finite Temperature}
\label{sec:lin_finite}
The finite temperature case needs some additional considerations as
the previous sections dealt explicitly with mean values in a pure
state. The Ehrenfest theorem still holds in mixed states, but we would
like to recover the usual pure state zero temperature formalism. There
are multiple methods for dealing with a thermal initial such as the
thermofield method (see~\cite{Diosi1998Mar}), but because the results
discussed here are to be applied with the HOPS method we shall use the
method described in~\cite{Hartmann2017Dec}.

The shift operator
\begin{equation}
  \label{eq:shiftop}
  \vb{D}(y) = \bigotimes_\lambda \eu^{y_\lambda a_\lambda^†-y^\ast_\lambda a_\lambda}
\end{equation}
the ground state of the environment into an arbitrary
coherent state
\begin{equation}
  \label{eq:shiftwork}
  \vb{D}(y)\ket{0} = \ket{y}
\end{equation}
where \(y=(y_1,y_2,\ldots)\) as usual.

This allows us to write the density matrix of the system with a
thermal initial bath as
\begin{equation}
  \label{eq:shiftbath}
  \rho =
  \prod_\lambda\qty(∫\dd[2]{y_\lambda}
  \frac{\eu^{-\abs{y_\lambda}^2\bar{n}_\lambda}}{\pi\bar{n}_\lambda})
  U(t)\vb{D}(y)\ketbra{\psi}\otimes\ketbra{0}\vb{D}(y)^† U(t)^†.
\end{equation}
The usual step is now to insert \(\id =\vb{D}(y)\vb{D}^†(y)\) to
arrive at a new time translation operator
\begin{equation}
  \label{eq:utilde}
  \tilde{U}(t) = \vb{D}^†(y)U(t)\vb{D}(y)
\end{equation}
and to interpret the integral in \cref{eq:shiftbath} in a monte-carlo
sense which leads to a stochastic contribution to the system Hamiltonian
\begin{equation}
  \label{eq:thermalh}
  H_{\mathrm{sys}}^{\mathrm{shift}}=L ξ^{*}(t)+L^{†} ξ(t)
\end{equation}
with the stochastic process
\begin{equation}
  \label{eq:xiproc}
  ξ(t):=∑_{\lambda} g_{\lambda} y_{\lambda} \eu^{-\mathrm{i} ω_{\lambda} t}
\end{equation}
with corresponding moments \(\mathcal{M}(ξ(t))=0=\mathcal{M}(ξ(t) ξ(s))\) and
\[
\mathcal{M}\left(ξ(t) ξ^{*}(s)\right)=\frac{1}{\pi} ∫_{0}^{∞} \mathrm{d} ω \bar{n}(\beta ω) J(ω) e^{-\mathrm{i} ω(t-s)}.
\]
Remember that we want to calculate
\begin{equation}
  \label{eq:whatreallymatters}
  \begin{aligned}
    \ev{L^†\dot{B}(t)} &= \tr[L^†\dot{B}(t)\rho(t)] \\
                       &=
                         \begin{aligned}
                           \prod_\lambda&\qty(∫\dd[2]{y_\lambda}
                                          \frac{\eu^{-\abs{y_\lambda}^2\bar{n}_\lambda}}{\pi\bar{n}_\lambda})\\
                                        &\tr[L^†\dot{B}(t)
                                          U(t)\vb{D}(y)\ketbra{\psi}\otimes\ketbra{0}\vb{D}(y)^† U(t)^†].
                         \end{aligned}
  \end{aligned}
\end{equation}
To recover the zero temperature formulation of this expectation value we
again insert a \(\id\), but have to commute \(\vb{D}(y)^†\) past
\(\dot{B}(t)\). This leads to the expression
\begin{equation}
  \label{eq:pureagain}
  \begin{aligned}
    \ev{L^†\dot{B}(t)} &=\prod_\lambda\qty(∫\dd[2]{y_\lambda}
                         \frac{\eu^{-\abs{y_\lambda}^2\bar{n}_\lambda}}{\pi\bar{n}_\lambda})\\
                       &\qquad\times\tr[
                         \begin{aligned}
                           L^†(\dot{B}(t) + \dot{ξ}(t))
                           \vb{D}^†(y) &U(t)\vb{D}(y)\ketbra{\psi}\\
                                       &\otimes\ketbra{0}\vb{D}^†(y)U(t)^†\vb{D}(y)
                         \end{aligned}
                         ] \\
                       &=\prod_\lambda
                       \qty(∫\dd[2]{y_\lambda}
                         \frac{\eu^{-\abs{y_\lambda}^2\bar{n}_\lambda}}{\pi\bar{n}_\lambda})\\
                       &\qquad\times\tr[L^†\qty{\dot{B}(t) + \dot{ξ}(t)}
                         \tilde{U}(t)\ketbra{\psi}\otimes\ketbra{0} \tilde{U}(t)^†].
  \end{aligned}
\end{equation}
which returns us to the zero temperature formalism with a transformed
Hamiltonian and the replacement
\begin{eqnarray}
  \label{eq:breplacement}
  B(t) \rightarrow B(t) + ξ(t)
\end{eqnarray}
which plausibly corresponds to the \(L^†\) part of \(H_\inter + H_{\mathrm{sys}}^{\mathrm{shift}}\).

The result of this section may be generalized in much the same way as
in \cref{sec:nonlin_flow} and leads to the replacement
\cref{sec:nonlin_flow} in the results of \cref{eq:breplacement}.

The appearance of \(\dot{ξ}(t)\) may cause concern. However, for twice
differentiable \(\mathcal{M}(ξ(t)ξ^\ast(s))\) the sample trajectories
are smooth.  Alternatively we can calculate
\begin{equation}
  \label{eq:gettingarounddot}
  \begin{aligned}
    \ev{\dot{H}_{\mathrm{sys}}^{\mathrm{shift}}} &=
    \dv{\ev{H_{\mathrm{sys}}^{\mathrm{shift}}}}{t} -
    \frac{1}{\iu}\qty(\ev{H_{\mathrm{sys}}^{\mathrm{shift}}H} -\ev{H
      H_{\mathrm{sys}}^{\mathrm{shift}}}) \\
    &=\dv{\ev{H_{\mathrm{sys}}^{\mathrm{shift}}}}{t} -
    \frac{1}{\iu}\ev{[H_{\mathrm{sys}}^{\mathrm{shift}}, H]} \\
    &=\dv{\ev{H_{\mathrm{sys}}^{\mathrm{shift}}}}{t} -
    \frac{1}{\iu}\ev{[H_{\mathrm{sys}}^{\mathrm{shift}}, H_\inter]}.
  \end{aligned}
\end{equation}
Now,
\begin{equation}
  \label{eq:hshcomm}
  [H_{\mathrm{sys}}^{\mathrm{shift}}, H_\inter] = ξ(t) [L^†, L]
  B(t)^† + ξ^\ast(t) [L, L^†] B
\end{equation}
and therefore
\begin{equation}
  \label{eq:finalex}
  \ev{[H_{\mathrm{sys}}^{\mathrm{shift}}, H_\inter]} = -i \mathcal{M}_{η^\ast}\mel{\psi}{ξ(t)^\ast[L,L^†]D_t}{\psi}.
\end{equation}
This is an expression that we can easily evaluate with the HOPS
method.

\section{General Collective Bath Observables}
\label{sec:general_obs}
The results obtained in \cref{sec:flow_lin,sec:nonlin_flow,sec:lin_finite}
can be generalized to calculate expectation values (and thus moments)
of arbitrary observables of the form
\begin{equation}
  \label{eq:collective_obs}
  O = f(B^†, B) = ∑_{α}F_α\qty(B^†)^{α_1}B^{α_2}
\end{equation}
where \(α\) is a multi-index, \(B\) is as in~\cref{eq:totalH} and the
\(F_α\) are general observables acting on the system only. Note that
\(f\) is already normal-ordered. We will restrict the discussion to
the case of a single bath, as the generalization to multiple baths is
straight-forward.

To evaluate \(\ev{O}\) we have to find the value of
\(\ev{\qty(B^†)^a B^b}\). This can be achieved by interjecting the
overcomplete coherent state basis as in \cref{eq:interactev} which
leads us to expressions in the form of
\(\mel{ψ}{\qty(B^†)^a}{z}\mel{z}{B^b}{ψ}\).

For zero temperature, we find following the procedures of
\cref{sec:flow_lin},
\begin{align}
    \label{eq:bmel}\mel{z}{B^b}{ψ} &= (-\iu D_t)^b\ket{ψ(η^\ast,t)}
                      = (-\iu)^b
                      ∑_{\abs{\vb{k}}=b}\binom{b}{\vb{k}} \iu^{\vb{k}}
                      \sqrt{\frac{G^{\vb{k}}}{\vb{k}!}}ψ^{\vb{k}}\\
    \label{eq:bdagmel}\mel{ψ}{\qty(B^†)^a}{z} &=
                              \begin{aligned}[t]
                                \qty(\mel{z}{B^a}{ψ})^†&= \qty((-\iu D_t)^b\ket{ψ(η^\ast,t)})^\dag\\
                                                   &= (\iu)^a∑_{\abs{\vb{k}}=a}\binom{a}{\vb{k}} (-\iu)^{\vb{k}}
                                                     \sqrt{\frac{\qty(G^{\vb{k}})^\ast}{\vb{k}!}}\qty(ψ^{\vb{k}})^\dag
                              \end{aligned}
\end{align}
in ``fock-space'' normalization where \(\vb{k}! = k_1!k_2!\ldots\) and
\(G^{\vb{k}}=G_1^{k_1}G_2^{k_2}\ldots\) following the usual
conventions of multi-indices. Thus, expressions involving the bath
operator \(B\) to the \(b\)th power lead to expressions involving the
hierarchy states of depth \(b\). The truncation of the hierarchy
corresponds to neglecting the expectation value of all powers of \(B\)
greater than the cutoff depth.

Returning to \cref{eq:collective_obs}, we find
\begin{equation}
  \label{eq:f_ex_zero}
  \begin{aligned}
  \ev{O} &= \mathcal{M}_{η^\ast}∑_{α} ∑_{\substack{\abs{γ}=α_1\\\abs{δ}=α_2}}
           \binom{α_1}{γ}\binom{α_2}{δ}(\iu)^{δ+α_1}(-\iu)^{γ+α_2}
           \sqrt{\frac{\qty(G^γ)^\ast G^δ}{γ!δ!}} \qty(ψ^γ)^\dag F_α ψ^δ.
  \end{aligned}
\end{equation}

For finite temperatures we substitute \(B(t)\to B(t)+ξ(t)\)
(interaction picture) as in
\cref{sec:lin_finite} to obtain
\begin{equation}
  \label{eq:finite_arb}
  \mel{z}{\qty(B(t)+ξ(t))^b}{ψ} = ∑_{m=0}^b \binom{b}{m} ξ^{b-m}(t)(-\iu D_t)^m\ket{ψ(η^\ast,t)}
\end{equation}
which may be substituted into the above.

The nonlinear method can be accommodated as in
\cref{sec:nonlin_flow}. For the expressions like~\cref{eq:f_ex_zero}
involving the HOPS hierarchy states this reduces to dividing by the
norm of the zeroth hierarchy state.

The generalization to multiple baths may be performed in the same
manner as will be discussed in \cref{sec:multibath}. This allows to
calculate the expectation value involving multiple bath operators
\(B^{(n)}\). Interestingly, the generalization of \cref{eq:bmel} to
multiple baths immediately links hierarchy states of the form
\(ψ^{\underline{e}_{i,k_i} + \underline{e}_{j,k_j}}\) (see
\cref{sec:multihops} for notation) with correlations between the baths
\(\ev{B_i(t)B_j(t)}\). Under the assumption that these correlations
remain small, one could argue to neglect them when performing
numerical calculations. Verifying this behavior would be an
interesting task for future work.

By anti-normal ordering the \(B, B^\dag\) in~\cref{eq:collective_obs}
and inserting the coherent state resolution of unity we find terms of
the form
\begin{equation}
  \label{eq:with_process}
  \mel{z}{\qty(B^\dag)^b}{ψ} \sim \qty(η^\ast(t))^b\ket{ψ(η^\ast,t)}.
\end{equation}
The corresponding version of~\cref{eq:f_ex_zero} would only depend on
the zeroth order state and the stochastic processes. It has been
observed that expressions involving the stochastic process directly
tend to converge slower. However this statement comes without
empirical proof and its verification may be left to future
study. Also, this alternative method could be used convergence and
consistency check, as expressions of the form~\cref{eq:with_process}
only involve the HOPS cutoff and the exponential expansion of the BCF
in an indirect manner.

\subsection{Interaction Energy}
\label{sec:intener}
A simple application of the formalism discussed
in~\cref{sec:general_obs} is the expectation value of the interaction
Hamiltonian.

For zero temperature and the nonlinear method we arrive at
\begin{equation}
  \label{eq:intexp}
  \ev{H_\inter} =
  -\i
  \mathcal{M}_{\tilde{η}^\ast}\frac{\mel{\psi(\tilde{η},t)}{L^†\tilde{D}_t}{\psi(\tilde{η}^\ast,t)}}{\braket{\psi(\tilde{η},t)}{\psi(\tilde{η}^\ast,t)}}
  + \cc.
\end{equation}
The expression for the linear method is obtained by
simply leaving out the normalization.

In HOPS terms \cref{eq:intexp} corresponds to
\begin{equation}
  \label{eq:interhops}
  \ev{H_\inter} =  ∑_\mu\sqrt{G_\mu}
  \mathcal{M}_{η^\ast}\bra{\psi^{(0)}(η,
    t)}L^†\ket{\psi^{\vb{e}_\mu}(η^\ast,t)} + \cc.
\end{equation}

For nonzero temperature an extra term
\begin{equation}
  \label{eq:interexptherm}
  \mathcal{M}_{\tilde{η}^\ast}\frac{\mel{\psi(\tilde{η},t)}{L^†ξ(t)}{\psi(\tilde{η}^\ast,t)}}{\braket{\psi(\tilde{η},t)}{\psi(\tilde{η}^\ast,t)}}
  + \cc
\end{equation}
has to be added to \cref{eq:intexp}, where \(ξ\) is the thermal
stochastic process.

\subsection{Higher Orders of the Coupling Hamiltonian}
\label{sec:higher_order_coupling}

For self adjoint coupling operators \(L=L^\dag\) we can use Wick's
theorem to find a normally ordered expression for \(H_\inter^n=L^n(B^\dag +
B)^n\).

The relevant contraction of \((B^\dag + B)(B^\dag + B)\) is
\begin{equation}
  \label{eq:contraction_b}
  (B^\dag + B)(B^\dag + B) - \mathopen{:} (B^\dag + B)(B^\dag + B)\mathclose{:} = α(0)
\end{equation}
with the (zero temperature) bath correlation function \(α\). In this
way, the interaction operator behaves like a generalized boson.

We find
\begin{equation}
  \label{eq:interactionnormal}
  \qty(H_\inter)^n = L^n ∑_{l=0}^{\lfloor {\frac{n}{2}} \rfloor}
  \frac{n!}{l!}{\qty(\frac{α(0)}{2})}^l
  ∑_{k=0}^{n-2l}\frac{1}{k!(n-2l-k)!}\qty(B^\dag)^k B^{n-2l-k},
\end{equation}
which can be further evaluated using \cref{eq:f_ex_zero}.

Notably,
the \(n\)th moment of \(H_\inter\) will depend on the hierarchy states
up to the depth \(n\). However, the terms involving similar exponents
for \(B^\dag\) and \(B\) will have the largest weights, so that the
hierarchy states of depth \(n/2\) play an important role. Truncating
the hierarchy will not simply lead to all higher orders moments of the
interaction being zero. Instead, the dependence upon the hierarchy
states is nontrivial. The explicit appearance of \(α(0)\) can be
avoided if we choose \(B\) unitless and \(α(0) = 1\).

Minimizing the coefficient \(l! 2^l k! (n-2l-k)!\) will give us the
order of \(B\) and thus the hierarchy depth that will contribute the
most to \cref{{eq:interactionnormal}}.  It suffices to minimize the
logarithm,
\begin{equation}
  \label{eq:logcoeff}
  Λ=\ln(2)l + \ln(Γ(l+1)) + \ln(Γ(k+1)) + \ln(Γ(n-2l -k + 1)).
\end{equation}

We begin by minimizing for \(k\)
\begin{equation}
  \label{eq:mink}
  \pdv{Λ}{k} = ψ(k+1)-ψ(n-2l-k+1) \overset{!}{=}0,
\end{equation}
where \(ψ\) is the Digamma function~\cite[p. 136]{Nisthb}.
As the Digamma function is strictly monotonic and thus injective,
\cref{eq:mink} implies \(k+1=n-2l-k+1\) and thus
\(l=n/2-k\). Note that the optimizing \(k\) is also the order of the
\(B\) operator associated with the coefficient.

Using this it remains to minimize
\begin{equation}
  \label{eq:logcoeff_k}
  Λ'=\ln(2)\qty(\frac{n}{2}-k) + \ln(Γ\qty(\frac{n}{2}-k+1)) + 2\ln(Γ(k+1)).
\end{equation}
yielding
\begin{equation}
  \label{eq:minkk}
  \begin{aligned}
    \pdv{Λ'}{k} &= 2ψ(k+1) -\ln(2) - ψ\qty(\qty(\frac{n}{2}-k+1))\\
                &\sim 2 \ln(k+1) - \ln(2) -\ln(\frac{n}{2}-k+1)\\
                &\overset{!}{=}0
  \end{aligned}
\end{equation}
and finally
\begin{equation}
  \label{eq:finalk}
  k_m=\left\lceil\sqrt{5+n}-2\right\rceil\sim\sqrt{n},
\end{equation}
where the ceiling has been chosen to obtain an integer. The error this
introduces is of the order of one.  We find that the ``most
important'' order of \(B\) required to calculate \(H_\inter^n\) scales
with \(\sqrt{n}\).  Note however, that the distribution of weights as
shown in~\cref{fig:kdist} will become broader with larger \(n\) and
that a specific \(k\) can occur multiple times, which will slightly
shift the maximum. A binomial (or Gaussian) distribution centered at
\(k_m\) as shown in \cref{fig:kdist} appears to be a good fit for the
relevant parameter space. Such a distribution has a standard deviation
of \(n^{{1}/{4}}\) for large \(n\). The position tail of the \(k\)
distribution thus scales like \(n^{1/2} + n^{1/4}\). The coupling
strength does only enter as a scale that controls which moment of
\(H_I\) is still relevant, albeit such a discussion should rather be
made for centered and normalized moments as they are
dimensionless. Without an a-priori estimate of \(\ev{H_I}\) and the
norm of the hierarchy states the above discussion has to be taken with
a grain of salt. Such estimates may possibly be obtained from the
estimates of the hierarchy state norms in \cref{sec:normest} and
contain not only the coupling strength, but also the shape of the BCF.

The result to take away is, that the expectation value of \(H_I^n\)
depends strongly on hierarchy states of order \(\sim \sqrt{n}\).

\begin{figure}[h]
  \centering
  \plot{interaction_orders/k_weights}
  \caption{\label{fig:kdist}The unnormalized distribution of the coefficients in
    \cref{eq:interactionnormal} with respect to \(k\) for different
    \(n\). As a particular \(k\) can appear multiple times in the sum,
    only the maximal coefficient for each \(k\) is being considered in
    the lines with the triangle markers. The maximum of this
    distribution is given by \cref{eq:finalk}. The lines with the
    circle markers show the full distribution. The dotted lines
    correspond to binomial distributions centered at \(k_m\).}
\end{figure}

\section{Multiple Baths}
\label{sec:multibath}
The results above can be generalized in straight-forward manner to
models of the form
\begin{equation}
  \label{eq:multimodel}
  H = H_\sys + ∑_{n=1}^N \qty[H_\bath\nth + \qty(L_n^†B_n + \hc)],
\end{equation}
where \(N\) is the number of baths, \(H_\sys\) is the (possibly time
dependent) system Hamiltonian,
\(H_\bath\nth = ∑_λω_λ\nth a_λ^{(n),†}a_λ\nth\),
\(B_n=∑_{λ} g_λ\nth a_λ\nth\) and the \(L_n={(\vb{L})}_n\) are
arbitrary operators in the system Hilbert space (again possibly time
dependent). This models a situation where each bath couples with the
system through exactly one spectral density and is therefore not fully
general.

We refer to \cref{sec:hops_multibath} for an review of the NMQSD
theory and HOPS method for multiple baths.

Because the bath energy change is being calculated directly and not
through energy conservation as in~\cite{Kato2016Dec}, we find
\begin{equation}
  \label{eq:general_n_flow}
  J_n=-\dv{\ev{H_\bath^{(n)}}}{t} = \iu\ev{[H_\bath^{(n)},
  H_\inter^{(n)}]}
\end{equation}
regardless of the (non-) commutativity\footnote{For example, the
  three-level model used in \cite{Uzdin2015Sep,Klatzow2019Mar} has
  non-commuting couplings.} of the interaction
Hamiltonians. Therefore, we can apply the formalism of the previous
sections almost unchanged, by taking care that all quantities involved
in the expression of \(J_n\) refer to the \(n\)th bath denoted by sub
and superscripts.

This can be achieved by making the replacements
\begin{equation}
  \label{eq:replacements}
  \begin{aligned}
    D_t \rightarrow D_t^{(n)} &\equiv
    ∫_0^t\dd{s}α_n(t-s)\fdv{η^\ast_n(s)} \\
    ξ(t) \rightarrow ξ_n(t)&\equiv∑_{\lambda} g^{(n)}_{\lambda}
    y_{\lambda} \eu^{-\mathrm{i} ω^{(n)}_{\lambda} t}
  \end{aligned}
\end{equation}
in the previous sections, where the quantities involved are as in
\cref{sec:hops_multibath} and \cref{eq:xiproc}.

In the light of \cref{sec:general_obs} it might be an interesting
question what impact mixed bath hierarchy states have. For a cyclic
machine with long strokes, where only one bath is coupled to the
system at a time, it might be efficient to truncate the hierarchy in a
way that discards mixed bath states more readily than single bath
hierarchy states as the correlations between the baths are expected to
be small.

\section{Time Dependent Hamiltonian}
\label{sec:timedep}
The above discussion is based on the model \cref{eq:totalH} which did
not include explicit time modulations of \(H_\sys\) or \(L\). As we
did not calculate any explicit time derivatives of those two
operators, the results of the previous sections remain valid when we
substitute
\begin{align}
  \label{eq:timedepsusbs}
  H_\sys&\rightarrow H_\sys(t) & L\rightarrow L(t).
\end{align}

For the total power we find
\begin{equation}
  \label{eq:power}
  \dv{\ev{H}}{t} = \ev{\pdv{H_I}{t}},
\end{equation}
which can be evaluated as in \cref{sec:intener} by replacing \(L(t)\)
with \(\dot{L}(t)\).

\section{Pure Dephasing: The initial Slip}
\label{sec:pure_deph}
As seen in \fixme{include plots}, the short time behavior of the bath
energy flow is dominated by characteristic peak at short
times. Because this peak occurs at very short time scales, it may in
part be explained by a simple calculation which neglects the system
dynamics, setting \(H_\sys=0\).

We solve the model with the Hamiltonian (Schr\"odinger picture)
\begin{equation}
  \label{eq:puredeph}
  H = L^†(t) B + L(t) B^† + H_\bath
\end{equation}
with \(L(t)=L(t)^†\), \([L(t), L(s)] = 0\;\forall t,s\) (so that
Heisenberg Hamiltonian matches \cref{eq:puredeph}) and \(B,H_\bath\)
as in \cref{eq:bop}.

Because \([L,H]=0\) we can immediately solve \(L_H(t)=L_S(t)\), where
the subscript signify the Heisenberg and Schr\"odinger pictures
respectively. The Heisenberg equations for the \(a_λ\) yield
\begin{equation}
  \label{eq:alapuredeph}
  a_λ(t) = a_λ(0) \eu^{-\iu ω_λ  t} - \iu g_λ^\ast∫_0^t\dd{s} L(s)
  \eu^{-\iu ω_λ  (t-s)}.
\end{equation}

This allows us to calculate
\begin{equation}
  \label{eq:pureflow}
  \dot{H}_\bath = - ∑_λ g_λ L(t) \qty[∂_t a_λ(0) \eu^{\iu ω_λ t} - \iu
  g_λ^\ast∫_0^t\dd{s} L(s) ∂_t \eu^{-\iu ω_λ (t-s)}] + \hc,
\end{equation}
which gives with a state of the form \(ρ=\ketbra{ψ} \otimes ρ_β\)
(\(ρ_β\) being a thermal state)
\begin{equation}
  \label{eq:pureflowexpectation}
  \ev{\dot{H}_\bath } = -2 ∫_0^t\dd{s}\ev{L(t)L(s)} \Im[\dot{α}(t-s)].
\end{equation}

For time independent \(L\) this becomes
\begin{equation}
  \label{eq:pureflowtimeindep}
  \ev{\dot{H}_\bath } = 2 \ev{L^2} \Im[\dot{α}(t)].
\end{equation}

The proportionality to the imaginary BCF \(α\) does explain the
initial peak in the bath energy flow. The imaginary part of the BCF is
zero for \(t=0\) and then usually features a peak at rather short
times (assuming finite correlation times). For the ohmic BCF used
here, this feature is very prominent.
\fixme{insert graph}

Interestingly, \cref{eq:pureflowexpectation} does not contain any
reference to the temperature of the bath. Therefore, the bath energy
can only surpass its initial value in this model, as the dynamics
match the zero temperature case in which the bath has minimal energy
in the initial state. A thermodynamically useful model should
therefore feature an significant system dynamics that do not commute
with the interaction or fast modulation so that the Hamiltonian does
not commute with itself at different times. The latter may induce
deviations from the pure-dephasing behavior at very short time scales
and thus be useful for finite power output. \fixme{here the plot with
  energy extraction would be good.} Coupling that is not self-adjoint
\fixme{plot} may also have this effect, but may be harder to
physically motivate. For the spin-boson system it is the result of the
random wave approximation, which however does not imply weak
coupling~\cite{Irish2007Oct}.

For completeness, the interaction energy is given by
\begin{equation}
  \label{eq:pureinter}
  H_\inter = L(t)\qty[∑_λg_λ\qty(a_λ(0)\eu^{-\i ω_λ t} - \i
  g^\ast_λ∫_0^t\dd{s} L(s) \eu^{\i ω_λ (t-s)})] + \hc,
\end{equation}
yielding
\begin{equation}
  \label{eq:pureinterexp}
  \ev{H_\inter} = 2 ∫_0^t\dd{s}\ev{L(t)L(s)} \Im[α(t-s)].
\end{equation}
\fixme{plots}

It is useful to normalize the BCF based on \cref{eq:pureinterexp}, so
that the pure interaction energy build-up in the initial slip is
canceled. To make the normalization independent of \(L(t)\),
we choose the normalization to be
\begin{equation}
  \label{eq:bcfnorm}
  \begin{aligned}
  \mathcal{N} &= 2 \abs{\frac{\max_t\norm{L(t)L^\dag(t)+\hc}}{\max_t{\norm{H(t)}}} ∫_0^∞ \Im[α_u(τ)]\dd{τ}}\\
    α(τ) &= α_u(τ)/\mathcal{N},
  \end{aligned}
\end{equation}
where \(α_u\) is some unnormalized BCF. This normalization has the
useful property, that it neutralizes any scaling in \(L\). Note that
here the convention in which \(α\) is dimensionless is used.

% this is not true
% imaginary part becomes proportional to the Dirac delta in the limit
% where typical cutoff frequency \(ω_c\rightarrow ∞\). The integral over
% the real part of \(α\) always gives zero if the spectral density obeys
% \(J(0) = 0\) and tends to exhibit fast oscillations and fast decay in
% the large-cutoff limit. For weak coupling, it may therefore be
% neglected. This constitutes the Markov limit mentioned in
% \cite{Strunz2001Habil}.

The Ohmic-type BCF is
\begin{equation}
  \label{eq:normohmic}
  α(τ)=\frac{ω_c  s }{ (\max_t\norm{H})(1+\iu ω_c τ)^{s+1}},
\end{equation}
in this normalization. Note however, that the norm of the Hamiltonian
is assumed to be unity in the simulations referred to in this
thesis. \fixme{maybe change}
