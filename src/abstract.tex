\paragraph{Abstract}

Motivated by questions of quantum thermodynamics, the possibility of
accessing bath related observables in a Hamiltonian system bath
model are explored.  Examples that will be explored within the
framework of the Non-Markovian Quantum State Diffusion (NMQSD) and its
numerical implementation, the Hierarchy of Pure States (HOPS) are the
expectation value of the interaction Hamiltonian and the change of the
bath energy expectation value.  It is shown that a consistent
computation of these observables is indeed possible through comparison
with an exact solution for quantum Brownian motion and by testing
energy conservation.  The developed formalism is subsequently applied
to study the energy transfer in the fundamental spin-boson model. Some
theoretical notions related to energy extraction from open systems
with thermal baths are highlighted and applied to guide the study of
energy extraction in a driven spin boson model and a quantum Otto
cycle.


\paragraph{Zusammenfassung}

Motiviert durch Fragen der Quanten-Thermodynamik wird die Möglichkeit
des Zugriffs auf badbezogene Observablen in einem Hamiltonschen
System-Bad Modell erforscht.  Beispiele, die im Rahmen der Rahmen der
Non-Markovian Quantum State Diffusion (NMQSD) und ihrer numerischen
Implementierung, der Hierarchy of Pure States (HOPS), untersucht
werden, sind der Erwartungswert des Wechselwirkungs-Hamiltonians und
die Änderung des Bad-Energie Erwartungswertes.  Durch durch Vergleich
mit einer exakten Lösung für ein quantenmechanisches Modell der
Brownschen Bewegung und durch Prüfung der Energieerhaltung wird
gezeigt, dass eine konsistente Berechnung dieser Observablen moeglich
ist.  Der entwickelte Formalismus wird anschließend angewandt um den
Energietransfer im fundamentalen Spin-Boson-Modell zu
untersuchen. Einige theoretische Begriffe im Zusammenhang mit der
Energiegewinnung aus offenen Systemen mit thermischen Bädern werden
vorgestellt und angewendet, um die Untersuchung der Energieextraktion
in einem getriebenen Spin-Boson-Modell und einem quantenmechanischen
Otto Prozess zu leiten.
