\paragraph{Abstract}

Motivated by questions of quantum thermodynamics, the possibilities of
accessing bath related observables such as the expectation value of
the interaction Hamiltonian and the change of the bath energy
expectation value within the framework of the Non-Markovian Quantum
State Diffusion (NMQSD) and its numerical implementation, the
Hierarchy of Pure States (HOPS), are explored. It is shown that a
consistent computation of these observables is indeed possible through
comparison with an exact solution for quantum Brownian motion and by
testing energy conservation.  The developed formalism is subsequently
applied to study the energy transfer in the fundamental spin-boson
model. Some theoretical notions related to energy extraction from open
systems with thermal baths are highlighted and applied to guide the
study of energy extraction in a driven spin boson model and a quantum
Otto cycle.


\paragraph{Zusammenfassung}


Motiviert durch Fragen der Quanten-Thermodynamik werden die
Möglichkeiten des Zugangs zu badbezogenen Observablen wie dem
Erwartungswert des des Wechselwirkungs-Hamiltonians und der Änderung
des Energieerwartungswertes des Bades im Rahmen der Non-Markovian
Quantum State Diffusion (NMQSD) und ihrer numerischen Implementierung,
der Hierarchy of Pure States (HOPS), untersucht.  Durch Vergleich mit
einer exakten Lösung für ein quantenmechanisches Modell der Brownschen
Bewegung und durch Prüfung der Energieerhaltung im falle des
Spin-Boson Modells, wird gezeigt dass eine konsistente Berechnung
dieser Observablen möglich ist.  Der entwickelte Formalismus wird
anschließend angewendet um den Energietransfer im Spin-Boson-Modell zu
untersuchen. Einige theoretische Themen im Zusammenhang mit der
Energiegewinnung aus offenen Systemen mit thermischen Bädern werden
aufgef\"uhrt und angewendet, um die Untersuchung der Energieextraktion
in einem getriebenen Spin-Boson-Modell und einem quantenmechanischen
Otto Prozess zu leiten.
